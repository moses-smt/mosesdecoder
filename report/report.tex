\documentclass[10pt]{report} 
\usepackage{epsf}
\usepackage{graphicx}
\usepackage{index}
\usepackage{varioref}
\usepackage{amsmath}
\usepackage{multirow}
\usepackage{theorem} % for examples
\usepackage{alltt}
\usepackage{epic,eepic}
\usepackage{boxedminipage}
\usepackage{fancybox}
\usepackage[square]{natbib}
\usepackage{ps4pdf}
\usepackage{picins} % pictures next to paragraphs, Ondrej's part
\PSforPDF{
\usepackage{dtree} % a dependency tree
\usepackage{pstricks}
\usepackage{pst-node}
\usepackage{pst-plot}
}

\makeindex
\theoremstyle{plain}

\begin{document}
\title{\vspace{-25mm}\LARGE {\bf Final Report}\\[2mm]
of the\\[2mm]
2006 Language Engineering Workshop\\[15mm]
{\huge \bf Open Source Toolkit\\[2mm]
\bf for Statistical Machine Translation:\\[5mm]
Factored Translation Models\\[2mm]
and Confusion Network Decoding}\\[10mm]
{\tt \Large http://www.clsp.jhu.edu/ws2006/groups/ossmt/}\\[2mm]
{\tt \Large http://www.statmt.org/moses/}\\[15mm]
Johns Hopkins University\\[2mm]
Center for Speech and Language Processing}
\author{\large Philipp Koehn,
Marcello Federico,
Wade Shen,
Nicola Bertoldi,\\
\large Ond\v{r}ej Bojar,
Chris Callison-Burch,
Brooke Cowan,\\
\large Chris Dyer,
Hieu Hoang,
Richard Zens,\\
\large Alexandra Constantin,
Christine Corbett Moran,
Evan Herbst}

\maketitle

\section*{Abstract}

\newpage
\section*{Acknowledgments}

\newpage
\section*{Team Members}

\begin{itemize}
\item Philipp Koehn, Team Leader, University of Edinburgh
\item Marcello Federico, Senior Researcher, ITC-IRST
\item Wade Shen, Senior Researcher, Lincoln Labs
\item Nicola Bertoldi, Senior Researcher, ITC-IRST
\item Chris Callison-Burch, Graduate Student, University of Edinburgh
\item Richard Zens, Graduate Student, Aachen University
\item Hieu Hoang, Graduate Student, University of Edinburgh
\item Brooke Cowan, Graduate Student, MIT
\item Ond\v{r}ej Bojar, Graduate Student, Charles University
\item Chris Dyer, Graduate Student, University of Maryland
\item Alexandra Constantin, Undergraduate Student, Williams College
\item Evan Herbst, Undergraduate Student, Cornell
\item Christine Corbett Moran, Undergraduate Student, MIT
\end{itemize}

\tableofcontents

\chapter{Introduction}
{\sc Philipp Koehn: overview of the goals of the workshop 2-4 pages}

\chapter{Factored Translation Models}
The current state-of-the-art approach to statistical machine translation, so-called phrase-based models, are limited to the mapping of small text chunks without any explicit use of linguistic information, may it be morphological, syntactic, or semantic. Such additional information has been shown to be valuable, by integrating it in pre-processing or post-processing steps. 

For instance, gains have been achieved by handling Arabic morphology through stemming or splitting off of affixes that typically translate into individual words in English. Another example is our earlier work on methods to reorder German input, so it is more similar to English output sentence order, which makes it more amendable to the phrase-based approach ({\sc cite}).

However, a tighter integration of linguistic information into the translation model is desirable for two reasons:
\begin{itemize}
\item Translation models that operate on more general representations, such as lemmas instead of surface forms of words, can draw on richer statistics and overcome the data sparseness problems caused by limited training data.
\item Many aspects of translation can be best explained on a morphological, syntactic, or semantic level. Having such information available to the translation model allows the direct modeling of these aspects. For instance: reordering at the sentence level is mostly driven by general syntactic principles, local agreement constraints show up in morphology, etc.
\end{itemize}

Therefore, we developed a framework for statistical translation models that tightly integrates additional information. Our framework is an extension of the phrase-based model. It adds additional annotation at the word level. A word in our framework is not anymore only a token, but a vector of factors that represent different levels of annotation.

\begin{center}
\includegraphics[scale=1]{factors.pdf}
\end{center}

Typical factors that we experimented with at this point include surface form, lemma, part-of-speech tag, morphological features such as gender, count and case, automatic word classes, true case forms of words, shallow syntactic tags, as well as dedicated factors to ensure agreement between syntactically related items. 

\section{Motivation}
One example to illustrate the short-comings of the traditional surface word approach in statistical machine translation is the poor handling of morphology. Each word form is treated as a token in itself. This means that the translation model treats, say, the word {\em house} completely independent of the word {\em houses}. Any instance of {\em house} in the training data does not add any knowledge to the translation of {\em houses}. 

In the extreme case, while the translation of {\em house} may be known to the model, the word {\em houses} may be unknown and the system will not be able to translate it. While this problem does not show up as strongly in English --- due to the very limited morphological production in English --- it does constitute a significant problem for morphologically rich languages such as Arabic, German, Czech, etc.

Thus, it may be preferably to model translation between morphologically rich languages on the level of lemmas, and thus pooling the evidence for different word forms that derive from a common lemma. In such a model, we would want to translate lemma and morphological information separately, and combine this information on the target side to generate the ultimate output surface words.

Such a model, which makes more efficient use of the translation lexicon, can be defined as a factored translation model. See below for an illustration of this model in our framework.

\begin{center}
\includegraphics[scale=1]{factored-morphgen-symmetric.pdf}
\end{center}

Note that while we illustrate the use of factored translation models on such a linguistically motivated example, our framework also applies to models that incorporate statistically defined word classes.

\section{Decomposition of Factored Translation}\label{sec:factored-decomposition}
The translation of the factored representation of source words into the factored representation of target words is broken up into a sequence of {\bf mapping steps} that either {\bf translate} input factors into output factors, or {\bf generate} additional target factors from existing target factors.

Recall the previous  of a factored model that translates using morphological analysis and generation. This model breaks up the translation process into the following steps:
\begin{itemize}
\item Translating of input lemmas into target lemmas
\item Translating of morphological and syntactic factors
\item Generating of surface forms given the lemma and linguistic factors
\end{itemize}

Factored translation models build on the phrase-based approach that breaks up the translation of a sentence in the translation of small text chunks (so-called phrases). This model implicitly defines a segmentation of the input and output sentences into such phrases. See an example below:

\begin{center}
\includegraphics[scale=1]{phrase-model-houses.pdf}
\end{center}

Our current implementation of factored translation models follows strictly the phrase-based approach, with the additional decomposition of phrase translation into a sequence of mapping steps. Since all mapping steps operate on the same phrase segmentation of the input and output sentence into phrase pairs, we call these {\bf synchronous factored models}. 

Let us now take a closer look at one example, the translation of the one-word phrase {\em h{\"a}user} into English. The representation of {\em h{\"a}user} in German is: surface-form {\em h{\"a}user}, lemma {\em haus}, part-of-speech {\em NN}, count {\em plural}, case {\em nominative}, gender {\em neutral}. 

Given the three mapping steps in our morphological analysis and generation model may provide the following applicable mappings:
\begin{itemize}
\item {\bf Translation:} Mapping lemmas
\begin{itemize}
\item {\em haus $\rightarrow$ house, home, building, shell}
\end{itemize}
\item {\bf Translation:} Mapping morphology
\begin{itemize}
\item {\em NN$|$plural-nominative-neutral $\rightarrow$ NN$|$plural, NN$|$singular} 
\end{itemize}
\item {\bf Generation:} Generating surface forms
\begin{itemize}
\item {\em house$|$NN$|$plural $\rightarrow$ houses} 
\item {\em house$|$NN$|$singular $\rightarrow$ house} 
\item {\em home$|$NN$|$plural $\rightarrow$ homes} 
\item {\em ...}
\end{itemize}
\end{itemize}

The German {\em haus$|$NN$|$plural$|$nominative$|$neutral} is expanded as follows:
\begin{itemize}
\item {\bf Translation:} Mapping lemmas\\
{\em \{ ?$|$house$|$?$|$?,$\;\;$ ?$|$home$|$?$|$?,$\;\;$ ?$|$building$|$?$|$?,$\;\;$ ?$|$shell$|$?$|$? \}}
\item {\bf Translation:} Mapping morphology\\
{\em \{ ?$|$house$|$NN$|$plural,$\;\;$ ?$|$home$|$NN$|$plural,$\;\;$ ?$|$building$|$NN$|$plural,$\;\;$ ?$|$shell$|$NN$|$plural,$\;\;$ ?$|$house$|$NN$|$singular,$\;\;$ ...~\}}
\item {\bf Generation:} Generating surface forms\\
{\em \{ houses$|$house$|$NN$|$plural,$\;\;$ homes$|$home$|$NN$|$plural,$\;\;$ buildings$|$building$|$NN$|$plural,$\;\;$ shells$|$shell$|$NN$|$plural,$\;\;$ house$|$house$|$NN$|$singular,$\;\;$ ...~\}}
\end{itemize}

\section{Statistical Modeling}
Factored translation models follow closely the statistical modeling methods used in phrase-based models. Each of the mapping steps is modeled by a feature function. This function is learned from the training data, resulting in translation tables and generation tables.

Phrase-based statistical translation models are acquired from word-aligned parallel corpora by extracting all phrase-pairs that are consistent with the word alignment. Given the set of extracted word pairs with counts, various scoring functions are estimated, such as conditional phrase translation probabilities based on relative frequency estimation.

Factored models are also acquired from word-aligned parallel corpora. The tables for translation steps are extracted in the same way as phrase translation tables. The tables for generation steps are estimated on the target side only (the word alignment plays no role here, and in fact additional monolingual data may be used). Multiple scoring functions may be used for generation and translation steps, we used in our experiments
\begin{itemize}
\item five scores for translation steps: conditional phrase translation probabilities in both direction (foreign to English and vice versa), lexical translation probabilities (foreign to English and vice versa), and phrase count;
\item two scores for generation steps: conditional generation probabilities in both directions (new target factors given existing target factors and vice versa).
\end{itemize}

As in phrase-based models, the different components of the model are combined in a log-linear model. In addition to traditional components --- language model, reordering model, word and phrase count, etc. --- each mapping steps forms a component with five (translation) or two (generation) features. The feature weights in the log-linear model are determined using a minimum error rate training method (cite Och, simplex).

\section{Efficient Decoding}
Compared to phrase-based models, the decomposition of the phrase translation into several mapping steps creates additional computational complexity. Instead of a simple table lookup to obtain the possible translation for an input phrase, now a sequence of such tables have to be consulted and their content combined.

Since all translation steps operate on the same segmentation, the {\bf expansion} of these mapping steps can be efficiently pre-computed prior to the heuristic beam search, and stored as translation options (recall the example in Section~\ref{sec:factored-decomposition}, where we carried out the expansion for one input phrase). This means that the fundamental search algorithm does not change. Only the scoring of hypothesis becomes slightly more complex.

However, we need to be careful about the combinatorial explosion of the number of translation options given a sequence of mapping steps. If one or many mapping steps result in a vast increase of (intermediate) expansions, this may be become unmanageable. We currently address this problem by early pruning of expansions, and limiting the number of translation options per input phrase to a maximum number, by default 50.

\section{Future Research}
%{\sc add ideas from Wade Shen, Hieu Hoang, Chris Dyer}
\subsection{Smarter search for multi-factored models}
Although factored translation models can be successfully used to
improve translation quality (in terms of {\sc bleu} score, as well
as other metrics, such as the rate of agreement errors in the output
text), initial experiments suggest two changes to the translation
model and decoding strategy that will enable more sophisticated
models (that take advantage of linguistically motivated
decomposition and generation processes, for example) and enable the
models to be applied in situations where the target language is
morphologically less complex (such as English).

\subsubsection{Shorter secondary spans}

One significant limiting factor in the performance of multi-factored
translation models is the due to the present requirement that
successive translation steps all translate identical source and
target spans.  If a compatible translation is not found for a
secondary translation step (either because hypotheses with
compatible factors were discarded earlier or because there is no
possible translation in the phrase table for the secondary
translation step), the hypothesis is abandoned. This has
considerable benefit from a computational perspective since it
constrains the search space for potential targets when translating
secondary factors.  However, it causes a number of significant
problems:
\begin{enumerate}
  \item In models where a secondary factor is both generated from another
target factor and translated from a source factor, any pruning
before both steps have completed runs the risk of producing not just
degraded output, but failing to find any adequate translation.
  \item Because a compatible translation must be found in secondary steps
for a translation hypothesis to survive, it is difficult to filter
secondary translation tables.  This results in very large tables
which are inefficient to load and have considerable memory overhead.
  \item When secondary translation steps fail and hypotheses are
abandoned, the model is forced to rely on shorter translation units
for the primary translation step. This is in direct conflict to the
potential benefits that can be gained by richer statistics.
\end{enumerate}

There are several possible ways that the exact-span match
requirement might be addressed. One solution that is computationally
tractable is to back off to shorter spans only in the event of a
failure to find any possible translation candidates during
subsequent translation steps.  The problem that arises is how the
spans established should be translated once multiple translation
units can be used.  Reordering within phrases is certainly quite
common.  These can be further constrained to either match alignments
that are suggested by the initial span.

\subsubsection{Translation-constrained generation with FSTs}

Currently when translation hypotheses are enumerated for a
particular span of the source sentence (the first step in the
translation process), each step in the mapping (whether translation
or generation) occurs serially and pruning occurs after each step.
Furthermore, multiple generation and translation steps frequently
target the same factor in a particular model (for example, a
target-side lemma may generate target-side part of speech
candidates, and source-side part of speech information may also be
translated into target-side part of speech sequences).  When the
same factor is generated in multiple mapping steps, they must all
converge or the hypothesis is abandoned.

The serial approach to computing translation options has two primary
drawbacks:

\begin{enumerate}
  \item Since a translation hypotheses is abandoned unless all steps in the
mapping succeed fully, it is often the case that many hypotheses
that survive pruning after one step are abandoned at a later step,
and that many that were pruned would have ended up being a
reasonable hypothesis.
  \item There are an exponential number of generation candidates available
for a given target span (where the exponent is the length of the
span and the base is the average number of targets from a given
source in the generation table).
\end{enumerate}

To mitigate both of these problems, it is possible to execute
generation and translation steps concurrently.  The generation table
can be formalized as a finite state transducer that maps between
factors on the target language side. The phrase table can be
formalized as a finite state transducer (FST) that maps between
source language factors to target language factors.  Thus all
devices that generate a given target language factor can be
conceived of as FSTs.  These FSTs can be composed with well-known
algorithms that will generally run much more efficiently (this is
exactly the method that is used to minimize the paths searched
through confusion networks: the source-side of the phrase table is
treated as a finite state automaton that is intersected with the
confusion network).

\subsubsection{Hybrid multi-factor and single-factor models}

A characteristic feature of natural languages is that elements of a
wide variety of sizes, from sub-word morphemes to complete
sentential units may be lexicalized.  The larger lexicalized units
(for example, idioms and "stock phrases") frequently exhibit
idiosyncratic- rather than compositional- meaning and are the bread
and butter of conventional phrase-based machine translation systems.
The phrase model can simply "memorize" the larger units and their
corresponding translations, which often tend to be idiosyncratic in
the target language.  This is arguably one of the significant
benefits of conventional phrase-based translation models since
mistranslating common stock phrases results is significantly
diminished fluency and understanding, and common evaluation metrics
assign a great deal of value to correctly translated stock phrases
(since they are, by definition, several words in length and tend to
exhibit relatively fixed word order).

Multi-factored models that analyze the source language in terms of
underlying lemmas, parts of speech, and morphological information,
and then translate these factors in a piecemeal fashion may actually
result in a system that performs less well on commonly occurring
lexicalized phrases. There are at least two reasons for this. First,
the process of lexicalization results in the retention of archaic or
otherwise unusual forms, possibly in unusual configurations.  Thus,
when these units are analyzed, they exhibit unusual morphological
features and parts of speech.  These unusual features introduce
significant sparseness in the sequence models in the target language
and reduce the overall probability that would be assigned to a
correct translation.  Second, single-factored translation models
generally have very good data on lexicalized phrases (since they
must, in order to be acquired by language learners as lexicalized
elements, occur with a reasonable frequency).  Therefore even if the
underlying linguistic phenomena are rather unusual, they are well
modeled in both translation models and target language models.
Moreover, if the target translation does contain unusual items,
these are more likely to occur in a very specific context, which
will generally decrease the net language model cost that would
otherwise be expected for infrequently occurring items.

To retain the benefits associated with multi-factored models
(generalization across inflected forms, making use of data with
richer statistics) but retaining the benefits of single-factored
"surface" translation models (better handling of stock phrases), a
more effective method would be to allow both a surface form based
single-factored model to propose hypotheses for various spans in a
sentence that would compete with hypotheses generated by a
multi-factored model.  Since multi-factored models consist of
different models with different scoring functions, the costs
associated with the two classes of hypotheses are not directly
comparable.  To mitigate this difficulty and establish a trading
relation between the two classes of hypotheses, a single-factor
penalty parameter will be introduced that can be tuned along with
the other parameters used in decoding.


\chapter{Confusion Network Decoding}
{\sc Marcello Federico and Richard Zens: cut and paste from your journal paper?}

\chapter{Open Source Toolkit}
\section{Overall design}
In developing the Moses decoder we were aware that the system should be open-sourced if it were to gain the support and interest from the machine translation community that we had hoped. There were already several proprietary decoders available which frustrated the community as the details of their algorithms could not be analysed or changed.
However, making the source freely available is not enough. The decoder must also advance the state of the art in machine translation to be of interest to other researchers. Its translation quality and runtime resource consumption must be comparable with the best available decoders. Also, as far as possible, it should be compatible with current systems which minimize the learning curve for people who wish to migrate to Moses. 
We therefore kept to the following principles when developing Moses: \\
\begin{itemize}	
 \item Accessibility 
 \item 	Easy to Maintain
 \item	Flexibility
 \item	Easy for distributed team development
 \item	Portability
\end{itemize}
The number of functionality added in the six weeks by every member of the team at the Johns Hopkins University workshop, as can be seen from the figures below, is evident that many of these design goals were met.

\begin{tabular}{|r|r|r|}
\hline
$ $&$ Before JHU workshop $&$ After JHU workshop $\\
\hline
Lines of code	& 9000	& 15,652\\
Number of classes	& 30	& 60\\
Lines of code attributed & & \\
to original developer	& 100\%	& 54\%\\
\hline
\end{tabular}

\begin{figure}[h]
\centering
\includegraphics[scale=1]{hieu-1}
\caption{Percentage of code contribute by each developer}
\end{figure}

By adding factored translation to conventional phrase based decoding we hope to incorporate linguistic information into the translation process in order to create a competitive system.\\

Resource consumption is of great importance to researchers as it often determine whether or not experiments can be run or what compromises needs to be taken. We therefore also benchmarked resource usage against another phrase-based decoder, Pharaoh, as well as other decoders, to ensure that they were comparable in like-for-like decoding.\\

It is essential that features can be easily added, changed or replace, and that the decoder can be used as a ‘toolkit’ in ways not originally envisaged. We followed strict object oriented methodology; all functionality was abstracted into classes which can be more readily changed and extended. For example, we have two implementations of single factor language models which can be used depending on the functionality and licensing terms required. Other implementations for very large and distributed LMs are in the pipeline and can easily be integrated into Moses. The framework also allows for factored LMs; a joint factor and skipping LM are currently available.\\
\begin{center}
\begin{figure}[h]
\centering
\includegraphics[scale=1]{hieu-2}
\caption{Language Model Framework}
\end{figure}
\end{center}
Another example is the extension of Moses to accept confusion networks as input. This also required changes to the decoding mechanism.\\

\begin{center}
\begin{figure}[h]
\centering
\includegraphics[scale=1]{hieu-3}
\caption{Input}
\end{figure}
\end{center}

\begin{center}
\begin{figure}[h]
\centering
\includegraphics[scale=0.8]{hieu-4}
\caption{Translation Option Collection}
\end{figure}
\end{center}
Nevertheless, there will be occasions when changes need to be made which are unforeseen and unprepared. In these cases, the coding practises and styles we instigated should help, ensuring that the source code is clear, modular and consistent to enable the developers to quickly assess the algorithms and dependencies of any classes or functions that they may need to change.\\

A major change was implemented when we decided to collect all the score keeping information and functionality into one place. That this was implemented relatively painlessly must be partly due to the clarity of the source code.\\

\begin{center}
\begin{figure}[h]
\centering
\includegraphics[scale=0.8]{hieu-5}
\caption{Scoring framework}
\end{figure}
\end{center}

The decoder is packaged as a library to enable users to more easily comply with the LGPL license. The library can also be embedded in other programs, for example a GUI front-end or an integrated speech to text translator.

\subsection{Entry Point to Moses library}

The main entry point to the library is the class\\
\\
\indent{\tt Manager}\\
\\
For each sentence or confusion network to be decoded, this class is instantiated and the following function called\\
\\
\indent{\tt ProcessSentence()}\\
\\
Its outline is shown below\\
\\
\begin{tt}
\indent CreateTranslationOptions()\\
\indent for each stack in m\_hypoStack\\
\indent \indent prune stack\\
\indent \indent for each hypothesis in stack\\
\indent \indent \indent ProcessOneHypothesis()\\
\end{tt}\\
Each contiguous word coverage (‘span’) of the source sentence is analysed in\\ 
\indent {\tt CreateTranslationOptions() }\\
\\
and translations are created for that span. Then each hypothesis in each stack is processed in a loop. This loop starts with the stack where nothing has been translated which has been initialised with one empty hypothesis.
\\
\subsection{Creating Translations for Spans}
The outline of the function \\
\\
\indent {\tt TranslationOptionCollection::CreateTranslationOptions()}\\
\\
is as follows:\\
\\
\begin{tt}
\indent	for each span of the source input\\
\indent \indent	CreateTranslationOptionsForRange(span)\\
\indent	ProcessUnknownWord()\\
\indent	Prune()\\
\indent	CalcFutureScoreMatrix()\\
\end{tt}
\\
A translation option is a pre-processed translation of the source span, taking into account all the translation and generation steps required. Translations options are created in\\
\\
\indent {\tt CreateTranslationOptionsForRange()}
\\
which is out follows\\
\\
\begin{tt}
\indent	ProcessInitialTranslation()\\
\indent	for every subequent decoding step\\
\indent \indent	if step is ‘Translation’\\
\indent \indent \indent	DecodeStepTranslation::Process()\\
\indent \indent else if step is ‘Generation’\\
\indent \indent \indent DecodeStepGeneration::Process()\\
\indent Store translation options for use by decoder\\
\end{tt}
\\
However, each decoding step, whether translation or generation, is a subclass of\\
\\
\indent {\tt DecodeStep}\\
\\
so that the correct Process() is selected by polymorphism rather than using if statements as outlined above.
\subsection{Unknown Word Processing}
After translation options have been created for all contiguous spans, some positions may not have any translation options which covers it. In these cases, CreateTranslationOptionsForRange() is called again but the table limits on phrase and generation tables are ignored. \\
If this still fails to cover the position, then a new target word is create by copying the string for each factor from the untranslatable source word, or the string ‘UNK’ if the source factor is null.\\
\begin{center}
\begin{tabular}{|c|c|c|}
\hline
Source Word & & New Target Word \\ \hline
Jimmy	& 	$\to$	& Jimmy\\
Proper Noun	& $\to$	& Proper Noun\\
-	& $\to$	& UNK\\
-	& $\to$	& UNK\\ \hline
\end{tabular}
\end{center}

This algorithm is suitable for proper nouns and numbers, which are one of the main causes of unknown words, but is incorrect for rare conjugation of source words which have not been seen in the training corpus. The algorithm also assumes that the factor set are the same for both source and target language, for instance, th list of POS tags are the same for source and target. This is clearly not the case for the majority of language pairs. Language dependent processing of unknown words, perhaps based on morphology. is a subject of debate for inclusion into Moses.\\
Unknown word processing is also dependent on the input type - either sentences or confusion networks. This is handled by polymorphism, the call stack is\\
\\
\begin{tt}
\indent	Base::ProcessUnknownWord()\\
\indent \indent	Inherited::ProcessUnknownWord(position)\\
\indent \indent \indent	Base::ProcessOneUnknownWord()\\
\end{tt}
where\\
\indent {\tt Inherited::ProcessUnknownWord(position)}\\
\\
is dependent on the input type.
\subsection{Scoring}
A class is created which inherits from\\
\\
\indent {\tt ScoreProducer}\\
\\
for each scoring model. Moses currently uses the following scoring models:\\
\\
\begin{center}
\begin{tabular}{|r|r|}
\hline
$ Scoring model $&$ Class $\\
\hline
Distortion	& DistortionScoreProducer\\
WordPenalty	& WordPenaltyProducer\\
Translation	& PhraseDictionary\\
Generation	& GenerationDictionary\\
LanguageModel	& LanguageModel\\
\hline
\end{tabular}\\
\end{center}

The scoring framework includes the classes \\
\\
\begin{tt}
\indent ScoreIndexManager\\
\indent	ScoreComponentCollection\\
\end{tt}
\\
which takes care of maintaining and combining the scores from the different models for each hypothesis.
\subsection{Hypothesis}
A hypothesis represents a complete or incomplete translation of the source. Its main properties are
\begin{center}
\begin{tabular}{|r|l|}
\hline
$ Variables $&$ $\\
\hline
m\_sourceCompleted & Which source words have already been translated\\
m\_currSourceWordsRange & Source span current being translated\\
m\_targetPhrase & Target phrase currently being used\\
m\_prevHypo & Pointer to preceding hypothesis that translated \\
 & the other words, not including m\_currSourceWordsRange\\
m\_scoreBreakdown & Scores of each scoring model\\
m\_arcList & List of equivalent hypothesis which have lower\\
 & score than current hypothesis\\
\hline
\end{tabular}\\
\end{center}

Hypothesis are created by calling the constructor with the preceding hypothesis and an appropriate translation option. The constructors have been wrapped with static functions, Create(), to make use of a memory pool of hypotheses for performance.\\
\\
Many of the functionality in the Hypothesis class are for scoring. The outline call stack for this is\\
\\
\begin{tt}
\indent CalcScore()\\
\indent	\indent	CalcDistortionScore()\\
\indent	\indent	CalcLMScore()\\
\indent	\indent	CalcFutureScore()\\
\end{tt}
\\
The Hypothesis class also contains functions for recombination with other hypotheses. Before a hypothesis is added to a decoding stack, it is compare to other other hypotheses on the stack. If they have translated the same source words and the last n-words for each target factor are the same (where n is determined by the language models on that factor), then only the best scoring hypothesis will be kept. The losing hypothesis may be used latter when generating the n-best list but it is otherwise not used for creating the best translation.\\
\\
In practise, language models often backoff to lower n-gram than the context words they are given. Where it is available, we use information on the backoff to more agressively recombine hypotheses, potentially speeding up the decoding.\\
\\
The hypothesis comparison is evaluated in \\
\\
\indent {\tt NGramCompare()}\\
\\
while the recombination is processed in the hypothesis stack class\\
\\
\indent {\tt HypothesisCollection::AddPrune()}\\
\\
and in the comparison functor class\\
\\
\indent {\tt HypothesisRecombinationOrderer}
\subsection{Phrase Tables}	
The main function of the phrase table is to look up target phrases give a source phrase, encapsulated in the function\\
\indent {\tt PhraseDictionary::GetTargetPhraseCollection()}\\
There are currently two implementation of the PhraseDictionary class\\
\begin{tabular}{|l|l|}
\hline
PhraseDictionaryMemory & Based on std::map. Phrase table loaded\\ 
 & completely and held in memory\\
PhraseDictionaryTreeAdaptor & Binarized phrase table held on disk and \\
 & loaded on demand.\\
\hline
\end{tabular}
\subsection{Command Line Interface}
The subproject, moses-cmd, is a user of the Moses library and provides an illustration on how the library functions should be called. It is licensed under a BSD license to enable other users to copy it source code for using the Moses library in their own application.\\
\\
However, since most researchers will be using a command line program for running experiments, it will remain the defacto Moses application for the time being.\\
\\
Apart from the main() function, there are two classes which inherites from the moses abstract class, InputOutput:\\
\\
\indent {\tt	IOCommandLine}\\
\indent {\tt	IOFile (inherites from IOCommandLine)}\\
\\
These implement the required functions to read and write input and output (sentences and confusion network inputs, target phrases and n-best lists) from standard io or files.





\section{Software Engineering Aspects}

\subsection{Regression Tests}
Moses includes a suite of regression tests designed to ensure that
behavior that has been previously determined to be correct does not
break as new functionality is added, bugs are fixed, or performance
improvements are made. The baseline behavior for the regression
testing is determined in three ways:
\begin{enumerate}
  \item Expected behavior based on off-line calculations (for example,
  given a small phrase table and sample input, one can work through
  the search space manually and compute the expected scores for a translation hypothesis).
  \item Expected values based on comparisons with other systems (for
  example, language modeling toolkits provide the ability to score
  a sentence.  Such a tool can be used to calculate the expected value of
  the language model score that will be produced by the decoder).
  \item Expected values based on previous behavior of the decoder (some output behavior
  is so complex that it is impractical or impossible to determine externally
  what the expected values are; however, it is reasonable to assume that localized bug-fixes,
  the addition of new functionality, or performance improvements should not impact existing
  behavior).
\end{enumerate}
The nature of statistical machine translation decoding makes
achieving substantial and adequate test coverage possible with
simple black-box testing.  Aggregate statistics on the number of
hypotheses generated, pruned, explored, as well as comparisons of
the exact costs and translations for certain sample sentences
provide ample evidence that the models and code that is utilized in
decoding is working adequately since these values tend to be highly
sensitive to even minor changes in behavior.

\subsubsection{How it works}
The test harness (invoked with the command \texttt{run-test-suite})
runs the decoder that is to be tested (specified to the script with
the \texttt{--decoder} command line option) with a series of
configuration files and translation inputs.  The output from the
decoder, which is written to \texttt{stdout} and \texttt{stderr}, is
post-processed by small scripts that pull out the data that is going
to be compared for testing purposes.  These values are compared with
the baseline and a summary is generated.

Timing information is also provided so that changes that have
serious performance implications can be identified as they are made.
This information is dependent on a variety of factors (system load,
disk speed), so it is only useful as a rough estimate.
\subsubsection{Versioning}
The code for the regression test suite is in the
\texttt{regression/tests} subdirectory of the Subversion repository.
 The inputs and expected values for each test case in the test suite are stored
together in \texttt{regression-tests/tests}.  The test suite is
versioned together with the source code for several reasons:
\begin{enumerate}
  \item As bugs are identified and fixed that effect existing behavior, the
testing code needs to be updated.
  \item As new functionality is added, testing code exercising this functionality needs to be
  added (see below for more details).
\end{enumerate}
By versioning the regression tests together with the source code, it
should be possible to minimize when developers need to worry about
expected test failures.

The data (language models, phrase tables, generation tables, etc.)
that is used by the individual test cases is versioned along with
the source code, but because of its size (currently about 60MB), it
is not stored in Subversion.  When test suite is run in a new
environment or one with an improper version of the test data, it
will fail and provide instructions for retrieving and installing the
proper version of the testing data (via HTTP) from the test data
repository, which is currently \texttt{http://statmt.org}.

\subsubsection{Making changes to existing tests}
As changes are made that effect the decoder's interface (output
format, command line interface, or configuration file format) and
bugs that effect existing decoder behavior are fixed, it will often
be necessary to update either the expected values, the scripts that
post-process the decoder output, or the configuration files.  These
files can be edited in the same manner as the rest of the source
code and should be submitted along with the corresponding code
changes.

If changes need to be made to the test data, a new tar-ball must be
generated that contains all of the test data for all regression
tests and submitted to the repository maintainer.  Once it is
available for download, the \texttt{TEST\_DATA\_VERSION} constant in
\texttt{MosesRegressionTesting.pm} can be incremented to point to
the new version.

\subsubsection{Adding regression tests}
As new functionality is incorporated into Moses, regression tests
should be added that guarantee that it will continue to be behave
properly as further changes are made.  Generally, testing new models
with multi-factored models is recommended since common
single-factored models exercise only a subset of the logic.

If new regression tests have new data dependencies, the test data
will need to be updated.  For more information on this workflow,
refer to the previous section.

\subsection{Accessability}
The source code for the Moses project is housed at Sourceforge.net
in a subversion repository.  The URL for the project is:

\begin{quote}
    \texttt{http://sourceforge.net/projects/mosesdecoder/}
\end{quote}

The source code is publicly accessible and in two ways:
\begin{enumerate}
  \item Pre-packaged tar-balls are available for download directly
  from project page at Sourceforge.
  \item The current development source code can be accessed with a subversion client (see \texttt{http://subversion.tigris.org/}
for more details how to acquire and use the client software).
\end{enumerate}


\subsection{Documentation}
{\sc Philipp Koehn and Chris Callison-Burch: Doxygen}

\section{Parallelization}
%{\sc Nicola Bertoldi}
The decoder implemented in {\tt Moses} translates its input sequentially; in order to increase 
the speed of the toolkit a parallelization module was developed which exploits several instances of the decoder and feed them with subsets of the scource input.


\begin{figure}
\begin{center}
\label{fig:parallel}
\caption{The parallelization module for {\tt Moses}.}
% \includegraphics[width=200mm]{Moses-parallel}
 \includegraphics[width=\columnwidth]{Moses-parallel}
\end{center}
 \end{figure}
 
As shown in Figure~\ref{fig:parallel}, the procedure we implemented is reasonably easy:
first, the source input is equally divided into $N$ parts, then $N$ instances of the {\tt Moses} translate them; finally, the full translation is obtained by ordering and merging the translation of all input parts.

All {\tt Moses} instances are assumed to run on a (possibly remote) cluster. No restriction on the number of {\tt Moses} instances is given.


Time to perform a full translation with one {\tt Moses} instance comprises the time to load data, which is constant, and time to translate the input, which is proportional to its size.
The parallelization module requires an additional time to access the cluster, which is strictly related to the real load of the cluster itself and hardly forecastable.
Time to split the input and merge the output can be considered negligible with respect to the translation time.
Moreover, an ending delay can be observe because the merging module should wait that all decoders have completed their translations, and this does not necessarily happen at the same time. A good splitting policy which allows a balanced translation time among all decoders,  improves the effciency of the whole parallelization module.

We tested the gain in time that  the parallelization module can provide to the toolkit on the Spanish-English EuroParl task. 3 input sets were created of 10, 100 1000 sentences and translated using a standalone  {\tt Moses}, and the parallelization module exploiting difference number of {\tt Moses} instances (1, 5, 10, 20).
Decoders ran on the 18-processor CLSP cluster. As in the real situation, its load was not in control, and hence the immediate availability of the processors was not assured. Table~\ref{tbl:parallel-time} reports the average translation times for all conditions.

Some considerations can be drawn by inspecting these figures.
\begin{itemize}
\item Parallelization is uneffective if source input is small, because time to access the cluster becomes prevalent.
\item Trivially, there is no reason of using the parallelization module if just one processor is required. 
\item Parallelization is beneficial if more instances of {\tt Moses} are exploited.
\item The gain in time is not exactly proportional to the number of decoder instances, mainly due to the effect  of ending delay.
\end{itemize}


\begin{table}
\label{tbl:parallel-time}
\begin{center}
\begin{tabular}{r|rrrrr}
                 & standalone   &1 proc  & 5 proc  &  10 proc   &  20 proc\\
                 \hline
     10 sentences   &  6.3  &  13.1 &  9.0  &  9.0   &     --  \\
  100 sentences   &  5.2  &  5.6    &  3.0  &  1.7   &   1.7 \\
1000 sentences   & 6.3   &  6.5    &  2.0  &  1.6   &   1.1 \\
\end{tabular}
\caption{Average time (seconds) to translate 3 input sets with a standalone {\tt Moses} and with its parallel version.}
\end{center}
\end{table}

In conclusion, the choice of the number of splits $N$ is essential for a good efficiency of the parallelization module, and depends on the available computational power, the cluster load, and the average translation time of the standalone decoder.



\section{Tuning}
%{\sc Nicola Bertoldi}

\newcommand{\e}{{\bf e}}
\newcommand{\f}{{\bf f}}
\label{sec:tuning}
As described in Section (reference to the correct section), {\tt Moses} decoder relies on a log-linear model to search for the best translation $\e^*$ given an input string $\f$:
\begin{equation}
\e^* = \arg\max_{\e}  \Pr(\e \mid \f) =\arg\max_{\e}  p_{\lambda}(\e \mid \f) = \arg\max_{\e} \sum_i \lambda_i h_i(\e,\f)
\end{equation}

Main components of a  log-linear model are the real-valued feature functions $h_i$ and their real-valued weights $\lambda_i$. To get the best performance from this model all components need to be estimated and optimized for the specific task the model is applied to.

Feature functions model specific aspects of the translation process, like the fluency, the adequacy, the reordering. Features can correspond to any function of $\e$ and $\f$, and  there is no restriction about the values they assume.
Some features are  based on statistical models which are estimated on specific training data.

Feature weights are useful to balance the (possibly very different) ranges of the feature functions, and to weigh their relative relevance. The most common way to estimate the weights of a log-linear model is called Minimum Error Rate Training (MERT). It consists in an automatic procedure which search for the weights minimizing translation errors on a development set.

Let $\f$ be a source sentence and ${\bf ref}$ the set of its reference translations; 
let $\bf Err(\e;ref)$ be an error function which measures the quality of a given translation $\e$ with respect to the references ${\bf ref}$. The MERT paradigm can be formally stated as follows:

\begin{eqnarray}
\e^*= \e^*(\lambda) = \arg\max_{\e}  p_{\lambda}(\e \mid \f) \\
\bf \lambda^* = \arg\min_{\lambda} {\bf Err} (\e^*(\lambda);ref)
\label{eq:directMT}
\end{eqnarray}
where $\e^*(\lambda)$ is the best translation found by the decoder exploiting a given set of weights $\lambda$.

The error function needs to be computed automatically from $\e$ and ${\bf ref}$ without human intervention. Word Error Rate (WER), Position Independent Word Error Rate (PER), (100-BLEU score), -NIST score, or any combination of them are good candidates as automatic scoring functions.

An error function is rarely mathematically sound, and hence an exact solution of the previous problem is not usually known. Hence, algorithms like the gradient descent or the downhill simplex, are exploited which iteratively approximate the optimal solution. Unfortunately, these approximate algorithms just assure to find a local optimum.

\begin{figure}
\begin{center}
\label{fig:MERT}
\caption{}
 \includegraphics[width=\columnwidth]{MER-ext}
\end{center}
 \end{figure}
 
The MERT procedure we implemented during the workshop is depicted in Figure~\ref{fig:MERT}. 
It is based on two nested loops, which are now described.

In the outer loop 
\begin{enumerate}
\item initial weights $\lambda^0$, an empty list of translation hypotheses $T^0$, and the iteration index $t=0$ are set;
\item {\tt Moses} translates the input with  $\lambda^t$ and generates a list of $N$-best translation hypotheses $T^t$;
\item $T^t$ are added to the previous lists $T^0, \ldots T^{t-1}$;
\item the inner loop is performed (see below) on the new list $\bigcup_{i=0}^{t}T^i$ and with the weights $\lambda^t$;
\item the  new set of weights $\lambda^{t+1}$ provided by the inner loop are set;
\item t is increased by 1, and the loop restarts from 2.
\end{enumerate}
The outer loop ends when the list of translation hypotheses does not increase anymore.

In the inner loop which  is fed with a list of hypotheses and a set of weights $\bar \lambda$
\begin{enumerate}
\item initial weights $\lambda^0$ are set to $\bar \lambda$, and the iteration index $s=0$ is set;
\item all translation hypotheses in the list are rescored according with the actual weights $\lambda^s$
and the best-scored hypothesis is extracted ({\tt Extractor});
\item the error measure of such translation is computed ({\tt Scorer});
\item the {\tt Optimizer} suggests a new set of weights $\lambda^{s+1}$;
\item $s$ is increased by 1, and the loop restarts from 2.
\end{enumerate}
The inner loop ends when the error measure does not improve anymore.
As the {\tt Optimizer} provides a local optimum for the weights, and strongly depends on the starting point $\bar \lambda$, the inner loop starts over several  times with different choices of $\bar \lambda$. The first time the weights $\lambda^t$ used by {\tt Moses} in the last outer loop are applied; the next times random sets are exploited. The best set of weights are then provided to the outer loop again.

Instead of standard approximate algorithms like the gradient descent or the downhill simplex, in the workshop we employed an {\tt Optimizer} which was developed by XXXX (should we cite the guy form UMaryland?) and based on the idea of \cite{xx} (Och's paper). The algorithm strongly relies on the availability of a finite list of translation alternatives, because this allows a discretization of the $r$-dimensional space of the weights ($r$ is the number of weights). This makes the search of the optimum  faster. The algorithm iteratively optimizes one weight at a time.

The {\tt Scorer} employed in the workshop computes BLEU score (version xxx).

The time spent for each iteration of the outer and inner loops is basically proportional to the size of the input and the amount of translation hypotheses, respectively.

\section{Efficient Language Model Handling}
{\sc Marcello Federico}

\section{Lexicalized Reordering Models}
{\sc Christine Moran}

\section{Error Analysis}
We describe some statistics generally used to measure error and present two error analysis tools written over the summer.

\subsection{Error Measurement}
There are three common measures of translation error. BiLingual Evaluation Understudy (BLEU) (\cite{bleu}), the most common, measures matches of short phrases between the translated and reference text as well as the difference in the lengths of the reference and output. BLEU can be applied to multiple references, but in a way such that BLEU scores using different numbers of references are not comparable.

Word Error Rate (WER) measures the number of matching output and reference words given that if output word $i$ is noted as matching reference word $j$, output word $i + 1$ cannot match any reference word before $j$; i.e., word ordering is preserved in both texts. Such a mapping isn't unique, so WER is specified using the maximum attainable number of single-word matches. This number is computable by some simple dynamic programming. [[[Ought I to elaborate here?]]]

Position-Independent Word Error Rate (PWER) simply counts matching output and reference words regardless of their order in the text. This allows for rearrangement of logical units of text, but allows a system to get away with poor rearrangement of function words.

All these measures are highly dependent on the level of redundancy in the target language: the more reasonable translation options, the less likely the one chosen will match the reference exactly. So the scores we use are really comparable only for a specific source text in a specific language.

Perplexity (defined in \cite{perplexity}), measured for a text with respect to a language model, is a function of the likelihood of that text being produced by repeated application of the model. In a shaky sense, he higher the perplexity of a text, the more complex it is, so the harder it is to produce. The perplexity of the output of a modern machine translation system is usually lower (for our test case, by a factor of two to three) than that of a reliable reference translation. This is unsurprising because the people who provide the references have at their command long-range syntactic constructs that haven't been reconstructed via computer.

Along with these statistics, we'd like some assurance that they're stable, preferably in the form of confidence intervals. We use both the paired $t$ test and the more conservative sign test to obtain confidence intervals for the BLEU score of each translation system on a corpus.

All of these measures can be applied to a text of any size, but the larger the text, the more statistical these scores become. For detail about the kinds of errors a translation system is making, we need sentence-by-sentence error analysis. For this purpose we wrote two graphical tools.

\subsection{Tools}
While working on his thesis Dr. Koehn wrote an online tool that keeps track of a set of corpuses (a corpus is a source text, at least one system output and at least one reference) and generates various statistics each time a corpus is added or changed. Before the workshop, his system showed BLEU scores and allowed a user to view individual sentences (source, output, reference) and score the output. For large numbers of sentences manual scoring isn't a good use of our time; the system was designed for small corpuses. To replace the manual-scoring feature we created a display of the BLEU scores in detail for each sentence: counts and graphical displays of matching n-grams of all sizes used by BLEU. See figure \ref{fig:sentence_by_sentence_screenshot} for screenshots.

The overall view for a corpus shows a list of files associated with a given corpus: a source text, one or more reference translations, one or more system translations. For the source it gives a count of unknown words in the source text (a measure of difficulty of translation, since we can't possibly correctly translate a word we don't recognize) and the perplexity. For each reference it shows perplexity. For each system output it shows WER and PWER, the difference between WER and PWER two for nouns and adjectives only (\cite{errMeasures}), the ratio of PWER of surface forms to PWER of lemmas (\cite{errMeasures}), and the results of some simple statistical tests, as described above, for the consistency of BLEU scores in different sections of the text. The system handles missing information decently, and shows the user a message to the effect that some measure is not computable. Also displayed are results of a $t$ test on BLEU scores between each pair of systems' outputs, which give the significance of the difference in BLEU scores of two systems on the same input.

\begin{figure}[h]
\centering
\caption{Sample output of corpus-statistics tool.}
\label{fig:sentence_by_sentence_screenshot}
%\subfloat[detailed view of sentences]{\frame{\vspace{.05in}\hspace{.05in}\includegraphics[width=6in]{}\hspace{.05in}\vspace{.05in}}} \newline
%\subfloat[overall corpus view]{\frame{\vspace{.05in}\hspace{.05in}\includegraphics[width=6in]{}\hspace{.05in}\vspace{.05in}}}
\end{figure}

A second tool developed during the workshop shows the mapping of individual source to output phrases (boxes of the same color on the two lines in figure \ref{fig:phrases_used_screenshot}) and gives the average source phrase length used. This statistic tells us how much use is being made of the translation model's capabilities. There's no need to take the time to tabulate all phrases of length 10, say, in the training source text if we're pretty sure that at translation time no source phrase longer than 4 words will be chosen.

\begin{figure}[h]
\centering
\caption{Sample output of phrase-detail tool.}
\label{fig:phrases_used_screenshot}
%\subfloat[]{\frame{\vspace{.05in}\hspace{.05in}\includegraphics[width=5in]{}\hspace{.05in}\vspace{.05in}}}
\end{figure}

%{\sc Evan Herbst}

\chapter{Experiments}

\section{English-German}
{\sc Philipp Koehn, Chris Callison-Burch, Chris Dyer}
\section{English-Spanish}
{\sc Wade Shen, Brooke Cowan, Christine Moran}

\section{English-Czech}
{  % wrapping all Ondrej's content to prevent confusing macros
%{\sc Ond\v{r}ej Bojar}


\def\clap#1{\hbox to 0pt{\hss #1\hss}}
\def\equo#1{``#1''}
\def\ang#1{{$\langle${#1}$\rangle$}}  % snadny zapis spicatych zavorek
\def\text#1{{\it{}#1}}


\def\bidir{Czech$\leftrightarrow$English}
\def\tocs{English$\rightarrow$Czech}
\def\toen{Czech$\rightarrow$English}
\def\parcite#1{\cite{#1}}
\def\perscite#1{\cite{#1}} % XXX no newcite available

\def\max#1{{\bf{} #1}}
\def\stddev#1{{$\pm$#1}}


\def\subsubsubsection#1{{\bf #1\\}}


This report describes in detail our experiments on \bidir{} translation
with the Moses system \parcite{moses} carried out at Johns Hopkins University
Summer Workshop 2006 in Baltimore. The reader is expected to be
familiar with factored translation models as implemented in Moses.

Section \ref{data} describes the data used for our experiments, including
preprocessing steps and some basic statistics. Section \ref{baselines}
introduces the metric and lists some known result on MT quality on our
dataset, including the scores of human translation. The core of this report is
contained in Section \ref{experiments} where all our experiments and results are
described in detail.




\subsection{Data Used}
\label{data}




\subsubsection{Corpus Description and Preprocessing}
\label{tools}


The data used for \bidir{} experiments are available as CzEng 0.5
\parcite{czeng:pbml:2006} and PCEDT 1.0 \parcite{pcedt:2004}. The collection contains parallel texts from various domains, as
summarized in Table~\ref{czengratios}.


\begin{table}[ht]
\begin{tabular}{lrr|rr}
  &  \multicolumn{2}{c}{Sentences}  &  \multicolumn{2}{c}{Tokens}\\
  &  Czech                             &  English                          &  Czech  &  English\\
\hline
Texts from European Parliament         &  77.7\%  &  71.7\%  &  78.2\%  &  75.9\%\\
E-Books                                &  5.5\%   &  6.6\%   &  7.2\%   &  7.4\%\\
KDE (open-source documentation)        &  6.9\%   &  10.2\%  &  2.6\%   &  3.6\%\\
PCEDT-WSJ (Wall Street Journal texts)  &  1.5\%   &  1.7\%   &  2.6\%   &  2.8\%\\
Reader's Digest stories                &  8.4\%   &  9.7\%   &  9.4\%   &  10.3\%\\
\hline
Total                                  &  1.4 M   &  1.3 M   &  19.0 M  &  21.6 M\\
\end{tabular}
\caption{Domains of texts contained in full training data.}
\label{czengratios}
\end{table}

The texts in CzEng are pre-tokenized and pre-segmented (sentence boundaries identified) and
automatically sentence-aligned using the hunalign tool
\parcite{varga:hunalign:2005}. The
PCEDT data are manually aligned 1-1 by origin, because the Czech
version of the text was obtained by translating English text sentence by
sentence.

For the purposes of our experiments, we processed the data using the tools
listed in Table \ref{toolsused}.
The English side of the corpus had to be retokenized (keeping CzEng sentence
boundaries), because the original tokenization was not compatible with the tagging tool.

\begin{table}[ht]
\begin{center}
\small
\begin{tabular}{lcc}
  &  Czech  &  English\\
\hline
Segmentation  &  CzEng  &  CzEng\\
Tokenization  &  CzEng  &  \clap{Like Europarl, \cite{koehn:europarl:mtsummit:2005}}\\
Morph./POS Tagging  &  \cite{hajhla:1998b}  &  \cite{mxpost:1996}\\
Lemmatization  &  \cite{hajhla:1998b}  &  -not used-\\
Parsing  &  \cite{mcdonald:pereira:ribarov:hajic:2005}  &  -not used-\\
\end{tabular}
\end{center}
\caption{Czech and English tools used to annotate CzEng data.}
\label{toolsused}
\end{table}



\subsubsection{Baseline (PCEDT) and Large (CzEng+PCEDT) Corpus Data}
\label{baselinelargecorpus}

The evaluation set of sentences used in our experiments (see section
\ref{references} below) comes from the very specific domain of Wall Street
Journal. The PCEDT-WSJ section matches this domain exactly, so we
use the PCEDT-WSJ section (20k sentences) as the training data in most of our experiments and refer
to it by the term \equo{baseline corpus} or simply PCEDT. In some
experiments, we make use of all the training data (860k sentences) and refer to it as the
\equo{large corpus}. (Of course, test data
are always excluded from training.)



\begin{table}[t]
\begin{center}
\begin{tabular}{llrrr}
Corpus  &    &  Sentences  &  Tokens\\
\hline
Baseline: PCEDT  &  Czech    &  20,581   &  453,050\\
                 &  English  &  20,581   &  474,336\\
\hline
Large: CzEng+PCEDT     &  Czech    &  862,398  &  10,657,552\\
                 &  English  &  862,398  &  12,001,772\\
\end{tabular}
\end{center}
\caption{Data sizes available for our experiments.}
\label{corpsizes}
\end{table}

Table \ref{corpsizes} reports exact data sizes of the baseline and large
corpora used for our experiments. (Note that the baseline corpus is a subset of
the large corpus.) The data size is significantly lower than what CzEng offers,
because not all of the sentences successfully passed through all our tools and
also due to the removal of sentences longer than 50 words and sentences with the ratio
between Czech and English number of tokens worse than 9.


% Including possible other data (licensing problems):
%File        	Sentences	Tokens
%forbidden.cs	1,030,872	12,339,260
%forbidden.en	1,030,872	13,894,186

%Data: /export/ws06osmt/data/cs-en/training 



\subsubsection{Tuning and Evaluation Data}
\label{references}


Our tuning and evaluation data consist of 515 sentences
with 4 reference
translations. The dataset was first published as part of PCEDT 1.0 for
evaluating \toen{} translation and included original English Wall Street
Journal (WSJ) texts translated to Czech (sentence by sentence) and 4 independent
back-translations to English. For the purposes of \tocs{} translation in our experiments, another 4
independent translations from the original English to Czech were obtained.

For our experiments we kept the original division of the dataset into two parts: the
tuning (development) set and the evaluation test set.
However, we retokenized all the sentences with the Europarl
tokenization tool. Dataset sizes in terms of number of sentences, input tokens
and input tokens never seen in the PCEDT training corpus (out-of-vocabulary,
OOV) are listed in Table \ref{tuneevaldata}.

\begin{table}[ht]
\begin{center}
\begin{tabular}{lccccc}
  &    &  \multicolumn{4}{c}{Input Tokens When Translating from}\\
   &  Sentences   &  Czech  &  OOV   &  English  &  OOV\\
\hline
Tuning   &  259  &  6429  &  6.8\%  &  6675  &  3.5\%\\
Evaluation  &  256  &  5980  &  6.9\%  &  6273  &  3.8\%\\
\end{tabular}
\end{center}
\caption{Tuning and evaluation data.}
\label{tuneevaldata}
\end{table}

We followed the common procedure to use tuning dataset to set parameters of the
translation system and to use the evaluation dataset for final translation
quality estimate. In other words, the translation system has access to the
reference translations of the tuning dataset but never has access to the
reference translations of the evaluation dataset.

In the following, we use the this short notation: \equo{Dev (std)} denotes
results obtained on the tuning dataset with the model parameters set to the
default, somewhat even distribution. \equo{Dev (opt)} denotes results on the
tuning dataset with the parameters optimized using minimum error rate training
procedure (MERT, XXX). The \equo{Dev (opt)} results are always overly
optimistic, because MERT had access to the reference translations and tunes the
MT output to get the highest scores possible. \equo{Test
(opt)} denotes results on evaluation set with model parameters as optimized on
the tuning set. The \equo{Test (opt)} results thus estimate the system
performance on unseen data and allow for a fair comparison.

For the purposes of automatic translation, the input texts were analyzed using
the same tools as listed in section \ref{tools}.



\subsection{MT Quality Metric and Known Baselines}
\label{baselines}

Throughout all our experiments, we use the BLEU metric \parcite{papineni:2002}
to automatically assess the quality of translation. We use an implementation of
this metric provided for the workshop. Other implementations such as IBM
original or NIST official {\tt mt\_eval} might give slightly
different absolute results, mainly due to different tokenization rules.

In all experiments reported below, we train and test the system in
\emph{case-insensitive} fashion (all data are converted to lowercase, including
the reference translations), except where stated otherwise.



\subsubsection{Human Cross-Evaluation}

Table \ref{crosseval} displays the scores if we evaluate one human translation
against 4 other human translations. For the sake of completeness, we report not
only the default lowercase (LC) evaluation but also case sensitive (CSens)
evaluation. This estimate cannot be understood as any kind of a bound or limit
on MT output scores, but it nevertheless gives some vague orientation when
reading BLEU figures.

\begin{table}[ht]
\begin{center}
\begin{tabular}{llccc|ccc}
   &     &  \multicolumn{3}{c}{To Czech}   &  \multicolumn{3}{c}{To English}\\
  &    &  Min  &  Average  &  Max  &  Min  &  Average  &  Max\\
\hline
Evaluation   &  LC   &  38.5  &  43.1$\pm$4.0  &  48.4   &  41.6  &  54.5$\pm$8.4  &  62.9\\
   &  CSens  &  38.1  &  42.5$\pm$4.0  &  47.8   &  41.1  &  53.8$\pm$8.4  &  62.4\\
\hline
Tuning   &  LC   &  39.0  &  46.3$\pm$4.3  &  49.3   &  45.8  &  55.3$\pm$6.0  &  61.7\\
   &  CSens  &  38.3  &  45.8$\pm$4.4  &  48.8   &  45.0  &  54.7$\pm$6.1  &  61.3\\
\end{tabular}
\end{center}
\caption{BLEU scores of human translation against 4 different human
translations. Evaluated 5 times, always comparing one translation against the 4
remaining. The minimum, average and maximum scores of the 5-fold estimation are
given.}
\label{crosseval}
\end{table}

As expected, we observe a higher variance (standard deviation) when evaluating
translation to English. The reason is that one of the five English versions of
the sentences is the original, while the other four were back translated
from Czech. It is therefore quite likely for the four back translations to differ
more from the original than from each other raising the BLEU variance.

English scores are generally higher and this may indicate that there is less variance
in word order, lexical selection or word morphology in English, but it also
could be the simple case that the translators to English produced more rigid
translations.


\subsubsection{BLEU When not Translating at All}

Our particular type of text (WSJ) contains a lot of numbers and proper names
that are often not altered during translation. Just for curiosity and to 
check that our datasets are not just numbers, punctuation and company names, we
evaluate BLEU for texts not translated at all. I.e. the input text is evaluated
against the standard 4 references. As displayed in Table
\ref{nontransl}, the scores are very low but nonzero, as expected.

\begin{table}[ht]
\begin{center}
\begin{tabular}{llcc}
   &     &  To Czech   &  To English\\
\hline
Evaluation   &  Lowercase   &  2.20  &  2.66\\
Evaluation   &  Case Sensitive   &  2.20  &  2.65\\
Tuning   &  Lowercase   &  2.93  &  3.60\\
Tuning   &  Case Sensitive   &  2.93  &  3.59\\
\end{tabular}
\end{center}
\caption{BLEU scores when not translating at all, i.e. only punctuation, numbers
and some proper names score.}
\label{nontransl}
\end{table}


\subsubsection{Previous Research Results}

Table \ref{comparison} summarizes previously published results of \toen{}
translation. Dependency-based MT (DBMT, \perscite{cmejrek:curin:havelka:2003})
is a system with rule-based transfer from Czech deep syntactic trees (obtained
automatically using one of two parsers of Czech) to English
syntactic trees. GIZA++ \parcite{och:ney:2003} and ReWrite
\parcite{rewrite:germann:2003} is the \equo{standard
baseline} word-based statistical system.
PBT \parcite{zens:etal:pbt:2005} is a phrase-based statistical
MT system developed at RWTH Aachen that has been evaluated on English-Czech data
by \perscite{bojar:etal:fintal:2006}.


\begin{table}[ht]
\begin{center}
\begin{tabular}{l@{\quad}c@{\quad}c@{\quad}|@{\quad}c@{\quad}c}
  &  \multicolumn{2}{@{\hspace{-1em}}c@{\quad}|@{\quad}}{Average over 5 refs.}  &  \multicolumn{2}{@{\quad}c}{4 refs. only}\\
%  &  \multicolumn{2}{c@{\quad}|@{\quad}}{Average  over four}  &  \multicolumn{2}{@{\quad}c@{\quad}}{Four re-translations}\\
%  &  \multicolumn{2}{c@{\quad}|@{\quad}}{re-translations + original}  &  \multicolumn{2}{@{\quad}c}{only}\\
  &  Dev  &  Test  &  Dev  &  Test\\
\hline
DBMT with parser I, no LM            &    18.57  &     16.34  &  -  &  -\\
DBMT with parser II, no LM           &    19.16  &     17.05  &  -  &  -\\
GIZA++ \& ReWrite, bigger LM         &    22.22  &     20.17  &  -  &  -\\
\hline                                   
PBT, no additional LM                &    38.7\stddev{1.5}  &  34.8\stddev{1.3}  &  36.3  &     32.5\\
PBT, bigger LM                       &    41.3\stddev{1.2}  &  36.4\stddev{1.3}  &  39.7  &     34.2\\
\hline                                                                                    
PBT, more parallel texts, bigger LM  &    42.3\stddev{1.1}  &  38.1\stddev{0.8}  &  41.0  &     36.8\\
\end{tabular}
\end{center}
\caption{Previously published results of \toen{} MT.}
\label{comparison}
\end{table}

All figures in Table \ref{comparison} are based on the same training dataset as
we use: the baseline corpus of PCEDT (20k sentences) and on the same tuning and
evaluation sets. However, the tokenization of the data is slightly different the
we use and also a different implementation of the BLEU metric was used. Our
experience is that a different scoring script can change BLEU results by about 2
points absolute, so these numbers should not be directly compared to our results
reported here.

Unlike \perscite{cmejrek:curin:havelka:2003} who evaluate
four-reference BLEU
five times using the original English text in addition to the 4 available
reference back-translations in a leave-one out procedure, we always report BLEU
estimated on the 4 reference translations only.


To the best of our knowledge, we are the first to evaluate \tocs{} machine
translation quality with automatic measures.




\subsection{Experiments}
\label{experiments}



\subsubsection{Motivation: Margin for Improving Morphology}
\label{margin}

Czech is a Slavonic language with very rich morphology and relatively free word
order. (See e.g. \perscite{bojar:cslp:2004} for more details.) The Czech
morphological system defines 4,000 tags in theory and 2,000 were actually seen
in a big tagged corpus. (For comparison, the English Penn Treebank tagset
contains just about 50 tags.) When translating to Czech, any MT system has to
face the richness and generate output words in appropriate forms.

Table \ref{morphmargin} displays BLEU scores of single-factored translation
\tocs{} using the baseline corpus only. The second line in the table gives the
scores if morphological information was disregarded in the evaluation: the MT
output is lemmatized (word forms replaced with their respective base forms) and evaluated against lemmatized references.

\begin{table}[ht]
\begin{center}
\begin{tabular}{lccc}
%     &  pcedt .t0-0. t0-0 LM0-3-pcedt  &                &            &  \\
  &  Dev (std)  &  Dev (opt)  &  Test (opt)\\
\hline
Regular BLEU, lowercase  &  25.68  &  29.24  &  25.23\\
Lemmatized MT output\\
\quad{}against lemmatized references  &  34.29  &  38.01  &  33.63\\
%margin:  &  8.61  &  8.77  &  8.40\\
\end{tabular}
\end{center}
\caption{Margin in BLEU for improving morphology.}
\label{morphmargin}
\end{table}

We see that more than 8 point BLEU absolute could be achieved if output word
forms were chosen correctly.\footnote{Although not all required word forms may
be available in the training data, we could easily generate output word forms
from lemmas and morphological tags deterministically using a large
target-side-only dictionary.} This observation gives us a strong motivation for
focussing on morphological errors first.



\subsubsection{Obtaining Reliable Word Alignment}

Given the richness of Czech morphological system and quite limited amount of
data in the baseline corpus (20k sentences), our first concern was to obtain
reliable word alignments. Like \perscite{bojar:etal:fintal:2006}, we reduce the
data sparseness by either lemmatizing or stemming Czech tokens and stemming
English tokens. (By stemming we mean truncating each word to at most 4
characters.) The vocabulary size of Czech word forms reduces to a half after
stemming or lemmatization and comes thus very close to the vocabulary size of English
word forms.

Table \ref{alignments} displays BLEU scores on Test (opt) \tocs{} depending on the
preprocessing of corpus for word alignment. The translation process itself was
performed on full word forms (single-factored), with a single trigram language
model collected from the Czech side of the parallel corpus. In all cases, we employed the
grow-diag-final heuristic for symmetrization of two independent GIZA++ runs.


\begin{table}[ht]
\begin{center}
\begin{tabular}{cccc}
\multicolumn{2}{c}{Preprocessing for Alignment}  &  \multicolumn{2}{c}{Parallel Corpus Used}\\
English          &  Czech       &  Baseline (20k sents.)  &  Large (860k sents.)\\
\hline
word forms       &  word forms  &                  25.17  &  -\\
4-char stems  &  lemmas      &                  25.23  &                25.40\\
4-char stems            &  4-char stems       &                  25.82  &                24.99\\
\end{tabular}
\end{center}
\caption{BLEU in \tocs{} translation depending on corpus preprocessing for word
alignment.}
\label{alignments}
\end{table}

The results confirm improvement in translation quality if we address the data
sparseness problem for alignments either by full lemmatization or by simple
stemming.  Surprisingly, using full lemmatization of the Czech side scored worse
than just stemming Czech. This result was confirmed neither on the large
training set, nor by \perscite{bojar:etal:fintal:2006} for \toen{} direction, so
we attribute this effect to random fluctuations in MERT procedure.

We also see nearly no gain or even some loss by increasing the corpus size
from 20k to 860k sentences. (See section \ref{moredata}
below for more details on various ways of using more data.)
This observation can be explained by the very specific
domain of our test set, see section


\subsubsection{Scenarios of Factored Translation \tocs{}}


\subsubsubsection{Scenarios Used}

We experimented with the following factored translation scenarios:

%\begin{tabular}{c@{\hspace{1cm}}c}
%English  &  Czech\\
%\hline
%\Rnode{elc}{lowercase}  &  \Rnode{clc}{lowercase}\\
%\end{tabular}
%\psset{nodesep=5pt,arrows=->}
%\everypsbox{\scriptstyle}
%\ncLine{elc}{clc}

\piccaptioninside
\piccaption{Single-factored scenario (T).}
\parpic[fr]{%
\PSforPDF{
\begin{tabular}{c@{}c@{\hspace{1cm}}c@{}c@{}c@{\hspace{5mm}}c}
English  &    &    &  Czech\\
\hline
lowercase  &  \Rnode{eAlc}{\strut}  &  \Rnode{cAlc}{}  &  {lowercase}  &  \Rnode{cAlcout}{}  &  +LM\\
morphology  &  \Rnode{eAtag}{\strut}  &  \Rnode{cAlemma}{}  &  {lemma} \Rnode{cAlemmaout}{}\\
  &    &  \Rnode{cAtag}{\strut}  &  morphology  &  \Rnode{cAtagout}{}\\
\end{tabular}
\psset{nodesep=1pt,arrows=->}
\everypsbox{\scriptstyle{}}
\ncLine{eAlc}{cAlc}
}
}

The baseline scenario is single-factored: input (English) lowercase word forms
are directly translated to target (Czech) lowercase forms. A 3-gram language
model (or more models based on various corpora) checks the stream of output word forms.

We call this the \equo{T} (translation) scenario.


\piccaption{Checking morphology (T+C).}
\parpic[fr]{%
\PSforPDF{
\begin{tabular}{c@{}c@{\hspace{1cm}}c@{}c@{}c@{\hspace{5mm}}c}
English  &    &    &  Czech\\
\hline
lowercase  &  \Rnode{eBlc}{\strut}  &  \Rnode{cBlc}{}  &  {lowercase}  &  \Rnode{cBlcout}{}  &  +LM\\
morphology  &  \Rnode{eBtag}{\strut}  &  \Rnode{cBlemma}{}  &  {lemma} \Rnode{cBlemmaout}{}\\
  &    &  \Rnode{cBtag}{\strut}  &  morphology  &  \Rnode{cBtagout}{}  &  +LM\\
\end{tabular}
\psset{nodesep=1pt,arrows=->}
\everypsbox{\scriptstyle{}}
\ncLine{eBlc}{cBlc}
\ncbar{->}{cBlcout}{cBtagout}
}
}


In order to check the output not only for word-level coherence but also
for morphological coherence, we add a single generation step: input word forms
are first translated to output word forms and each output word form then
generates its morphological tag.

Two types of language models can be used simultaneously: a (3-gram) LM over word forms and a
LM over morphological tags. For the morphological tags, a higher-order LM can be
used, such as 7 or 9-gram.

We used tags with various levels of detail, see section \ref{posgranularity}.
We call this the \equo{T+C} (translate and check) scenario.


\piccaption{Translating and checking morphology (T+T+C).}
\parpic[fr]{
\PSforPDF{
\begin{tabular}{c@{}c@{\hspace{1cm}}c@{}c@{}c@{\hspace{5mm}}c}
English  &    &    &  Czech\\
\hline
lowercase  &  \Rnode{eClc}{\strut}  &  \Rnode{cClc}{}  &  {lowercase}  &  \Rnode{cClcout}{}  &  +LM\\
morphology  &  \Rnode{eCtag}{\strut}  &  \Rnode{cClemma}{}  &  {lemma} \Rnode{cClemmaout}{}\\
  &    &  \Rnode{cCtag}{\strut}  &  morphology  &  \Rnode{cCtagout}{}  &  +LM\\
\end{tabular}
\psset{nodesep=1pt,arrows=->}
\everypsbox{\scriptstyle{}}
\ncLine{eClc}{cClc}
\ncLine{eCtag}{cCtag}
\ncbar{->}{cClcout}{cCtagout}
}
}

As a refinement of T+C, we also used T+T+C scenario, where the morphological
output stream is constructed based on both, output word forms and input
morphology. This setting should ensure correct translation of morphological
features such as number of source noun phrases.
%, while the T+C setting simply guessed number of noun
%phrases based on the language models.

Again, two types of language models can be used in this \equo{T+T+C} scenario.

\piccaption{Generating forms from lemmas and tags (T+T+G).}
\parpic[fr]{%
\PSforPDF{
\begin{tabular}{c@{}c@{\hspace{1cm}}c@{}c@{}c@{\hspace{5mm}}c}
English  &    &    &  Czech\\
\hline
lowercase  &  \Rnode{e2lc}{\strut}  &  \Rnode{c2lc}{}  &  {lowercase}  &  \Rnode{c2lcout}{}  &  +LM\\
morphology  &  \Rnode{e2tag}{\strut}  &  \Rnode{c2lemma}{}  &  {lemma} \Rnode{c2lemmaout}{}  &    &  +LM\\
  &    &  \Rnode{c2tag}{\strut}  &  morphology  &  \Rnode{c2tagout}{}  &  +LM\\
\end{tabular}
\psset{nodesep=1pt,arrows=->}
\everypsbox{\scriptstyle{}}
\ncLine{e2tag}{c2tag}
\ncLine{e2lc}{c2lemma}
\ncbar{->}{c2lemmaout}{c2lcout}
\ncbar{->}{c2tagout}{c2lcout}
}
}

The most complex scenario we used is linguistically appealing: output lemmas
(base forms) and morphological tags are generated from input in two independent
translation steps and combined in a single generation step to produce
output word forms. The input English text was not lemmatized so we used English
word forms as the source for producing Czech lemmas.

The \equo{T+T+G} setting allows us to use up to three types of language models.
Trigram models are used for word forms and lemmas and 7 or 9-gram language
models are used over tags.



%% small pictures
%\psset{unit=5mm}
%\begin{pspicture}(0,-0.5)(3,0.5)
%\psline{->}(0,0)(3,0)
%\pscircle*[fillcolor=white](0,0){0.5ex}
%\end{pspicture}
%
%\psset{unit=5mm}
%\begin{pspicture}(0,-0.5)(3,1.5)
%\psline{->}(0,1)(3,1)
%\psline{->}(3,1)(3,0)
%\end{pspicture}
%
%\psset{unit=5mm}
%\begin{pspicture}(0,-0.5)(3,2.5)
%\psline{->}(0,2)(3,2)
%\psline{->}(0,1)(3,0)
%\psline{->}(3,2)(3,0)
%\end{pspicture}
%
%
%\psset{unit=5mm}
%\begin{pspicture}(0,-0.5)(3,2.5)
%\psline{->}(0,2)(3,1)
%\psline{->}(0,1)(3,0)
%\psline{->}(3,1)(3.5,1)(3.5,2)(3,2)
%\psline{->}(3,0)(3.5,0)(3.5,2)(3,2)
%\end{pspicture}



\subsubsubsection{Experimental Results: T+C Works Best}

Table \ref{scenariosresults} summarizes estimated translation quality of the
various scenarios. In all experiments, only the baseline corpus of 20k sentences
was used with word alignment obtained using grow-diag-final heuristic on stemmed
English and lemmatized Czech input streams. Language models are also based
on the 20k sentences only, 3-grams are used for word forms and lemmas, 7-grams
for morphological tags.


\begin{table}[ht]
\begin{center}
\begin{tabular}{lccc}
  &  Dev (std)  &  Dev (opt)  &  Test (opt)\\
\hline
Baseline: T      &  25.68  &  29.24  &  25.23\\
T+T+G  &  23.93  &  30.34  &  25.94\\
T+T+C  &  25.12  &  30.73  &  26.43\\
T+C    &  23.51  &  \max{30.88}  &  \max{27.23}\\
\end{tabular}
\end{center}
\caption{BLEU scores of various translation scenarios.}
\label{scenariosresults}
\end{table}

The good news is that multi-factored models always outperform the baseline T
(except for \equo{Dev (std)}, but this is not surprising, as the default weights
can be quite bad for multi-factored translation).

Unfortunately, the more complex a multi-factored scenario is, the worse
the results are. Our belief is that this effect is caused by search errors: with
multi-factored models, more hypotheses get similar scores and future costs
of partial hypotheses might be estimated less reliably. With the limited stack
size (not more than 200 hypotheses of the same number of covered input words),
the decoder may more often find sub-optimal solutions.

To conclude, the scenario for just checking output morphology (T+C) gives us the
best results, 27.23 BLEU, 2 points absolute improvement over the single-factored
baseline.



\subsubsection{Granularity of Czech Part-of-Speech}
\label{posgranularity}


As stated above, Czech morphological tag system is very complex, in theory up to
4,000 different tags are possible. In our T+C scenario, we experiment with
various simplifications of the system to find the best balance between
expresivity and richness of the statistics available in our corpus. (The more
information is retained in the tags, the more severe data sparseness is.)

% Details in morftaginfo/*.info
%pcedt.cs.2.info	1                  	1098  	453050	412.61
%pcedt.cs.cng02.info	1                  	707   	453050	640.81
%pcedt.cs.cng03.info	1                  	899   	453050	503.95
%pcedt.cs.cng.info	1                  	571   	453050	793.43
%pcedt.cs.pos.info	1                  	173   	453050	2618.79


\begin{description}

\item[Full tags (1098 unique seen in baseline corpus):]
Full Czech positional tags are used. A tag consists of 15
positions, each holding the value of a morphological property (e.g. number, case
or gender).


\item[POS (173 unique seen):]
We simplify the tag to include only part and subpart of speech (distinguishes
also partially e.g. verb tenses). For nouns, pronouns, adjectives and
prepositions\footnotemark{}, also the case is included.\footnotetext{Some Czech prepositions select
for a particular case, some are ambiguous. Although the case is never shown on
surface of the preposition, the tagset includes this information and Czech
taggers are able to infer the case.}


\item[CNG01 (571 unique seen):]
CNG01 refines POS. For nouns, pronouns and adjectives we include not only the
case but also number and gender.


\item[CNG02 (707 unique seen):]
Tag for punctuation is refined: lemma of the punctuation symbol is taken into
  account; previous models disregarded e.g. the distributional differences between a comma and a
  question mark.
Case, number and gender added to nouns, pronouns, adjectives, prepositions,
  but also to verbs and numerals (where applicable).


\item[CNG03 (899 unique seen):]
Highly optimized tagset:
\begin{itemize}

\item Tags for nouns, adjectives, pronouns and numerals describe the case, number
  and gender; the Czech reflexive pronoun \text{se} or \text{si} is highlighted
  by a special flag.

\item Tag for verbs describes subpart of speech, number, gender, tense and
  aspect; the tag includes a special flag if the verb was the auxiliary verb
  \text{b\'{y}t (to be)} in any of its forms.

\item Tag for prepositions includes the case and also the lemma of the preposition.

\item Lemma included for punctuation, particles and interjections.

\item Tag for numbers describes the \equo{shape} of the number (all digits are
  replaced by the digit \text{5} but number-internal punctuation is kept
  intact). The tag thus distinguishes between 4- or 5-digit numbers or the
  precision of floating point numbers.

\item Part of speech and subpart of speech for all other words.
\end{itemize}

\end{description}


\subsubsubsection{Experimental Results: CNG03 Best}

Table \ref{granularityresults} summarizes the results of T+C scenario with
varying detail in morphological tag. All the results were obtained using only
the baseline corpus of 20k sentences, word-alignment symmetrized with
grow-diag-final heuristic and based on stemmed Czech and English input. Also the
language models are based solely on the 20k sentences. Trigrams are used for
word forms and 7-grams for tags.

\begin{table}[ht]
\begin{center}
\begin{tabular}{lccc}
  &  Dev (std)  &  Dev (opt)  &  Test (opt)\\
\hline
Baseline: T (single-factor)  &  26.52        &  28.77        &  25.82\\
T+C, CNG01     &  22.30         &  29.86        &  26.14\\
T+C, POS       &  21.77         &  30.27        &  26.57\\
T+C, full tags       &  22.56         &  29.65        &  27.04\\
T+C, CNG02     &  23.17         &  \max{30.77}        &  27.45\\
T+C, CNG03     &  23.27         &  30.75        &  \max{27.62}\\
\end{tabular}
\end{center}
\caption{BLEU scores of various granularities of morphological tags in T+C
scenario.}
\label{granularityresults}
\end{table}


Our results confirm significant improvement over single-factored baseline.
Detailed knowledge of the morphological systems also proves its utility: by
choosing the most relevant features of tags and lemmas but avoiding sparseness,
we can improve about 0.5 BLEU absolute over T+C with full tags. Too strict
reduction of features used causes a loss.



\subsubsection{More Out-of-Domain Data in T and T+C Scenarios}
\label{moredata}


%                                                           	BLEU.dev.opt	BLEU.dev.std	BLEU.opt	Class	Data
%czeng LM0-3-czeng t0-0                                     	       28.74	       23.47	   24.99	1	mixed
%pcedt LM0-3-pcedt t0-0                                     	       28.77	       26.52	   25.82	1	small
%czeng LM0-3-czeng LM1-7-czeng tag                          	       29.50	       15.66	   26.54	2	mixed
%pcedt LM0-3-pcedt LM1-7-pcedt tag                          	       29.65	       22.56	   27.04	2	small
%pcedt LM0-3-pcedt LM0-3-czeng LM1-7-pcedt LM1-7-czeng tag  	       30.33	       19.97	   27.15	2	separate
%pcedt LM0-3-pcedt LM0-3-czeng LM1-7-pcedt LM1-7-czeng cng03	       30.48	       19.95	   27.15	3	separate
%czeng LM0-3-czeng LM1-7-czeng cng03                        	       30.71	       15.77	   27.29	3	mixed
%czeng LM0-3-czeng LM0-3-pcedt t0-0                         	       29.42	       19.45	   27.41	1	separate
%czeng LM0-3-pcedt LM0-3-czeng LM1-7-pcedt LM1-7-czeng cng03	       30.33	       14.22	   27.48	3	separate
%pcedt LM0-3-pcedt LM1-7-pcedt cng03                        	       30.75	       23.27	   27.62	3	small
%czeng LM0-3-pcedt LM0-3-czeng LM1-7-pcedt LM1-7-czeng tag  	       30.64	       14.06	   28.12	2	separate

% definice trid:
% 1  ... t0-0
% 2  ... P+K tag
% 3  ... P+K cng03
% 4  ... P

%   !dett | sed 's/\%//' | tabrecalc '\qdisk(COL4,COL5){3pt}  \uput[ur](COL4,COL5){COL4 COL6}'


\begin{figure}[t]
\begin{center}
\PSforPDF{
\psset{xunit=25mm,yunit=10mm}
\begin{pspicture*}(23.5,-0.7)(28.5,3)
%\psgrid

\qdisk(24.99,0.5){2pt}  \uput[u](24.99,0.5){mix}
\qdisk(25.82,0.5){2pt}  \uput[u](25.82,0.5){s}
\qdisk(26.54,1.5){2pt}  \uput[u](26.54,1.5){mix}
\qdisk(27.04,1.5){2pt}  \uput[d](27.04,1.5){s}
\qdisk(27.15,1.5){2pt}  \uput[ur](27.15,1.5){s$^+$}
\qdisk(27.15,2.5){2pt}  \uput[u](27.15,2.5){s$^+$}
\qdisk(27.29,2.5){2pt}  \uput[d](27.29,2.5){mix}
\qdisk(27.41,0.5){2pt}  \uput[u](27.41,0.5){L}
\qdisk(27.48,2.5){2pt}  \uput[u](27.48,2.5){L}
\qdisk(27.62,2.5){2pt}  \uput[d](27.62,2.5){s}
\qdisk(28.12,1.5){2pt}  \uput[u](28.12,1.5){L}

\psline(23,1)(29,1)
\psline(23,2)(29,2)

\rput[l](23.5,0.5){T}
\rput[l](23.5,1.5){T+C full tags}
\rput[l](23.5,2.5){T+C CNG03}

\psaxes[Ox=23,Dx=1,Dy=5](23,0)(29,3)
\end{pspicture*}
}
\end{center}

\begin{center}
\small
\begin{tabular}{cp{0.85\textwidth}}
s  &  small data, 20k sentences in the domain\\
s$^+$  &  small data, 20k sentences in the domain with an additional separate LM (860k sents out of the domain)\\
L  &  large data, 860k sentences, separate in-domain and out-of-domain LMs\\
mix  &  large data, 860k sentences, a single LM mixes the domains\\
\end{tabular}
\end{center}
%\begin{center}
\footnotesize
\hspace{-10mm}
\begin{tabular}{lcccrrr}
Scenario  &  Acronym  &  Parallel Corpus  &  Language Models  &  Dev (std)  &  Dev (opt)  &  Test (opt)\\
\hline
T              &  mix    &  Large (860k)    &  Large (860k)    &  23.47  &  28.74  &        24.99\\
T              &  s      &  Baseline (20k)  &  Baseline (20k)  &  26.52  &  28.77  &        25.82\\
T+C full tags  &  mix    &  Large (860k)    &  Large (860k)    &  15.66  &  29.50  &        26.54\\
T+C full tags  &  s      &  Baseline (20k)  &  Baseline (20k)  &  22.56  &  29.65  &        27.04\\
T+C full tags  &  s$^+$  &  Baseline (20k)  &  20k+860k        &  19.97  &  30.33  &        27.15\\
T+C CNG03      &  s$^+$  &  Baseline (20k)  &  20k+860k        &  19.95  &  30.48  &        27.15\\
T+C CNG03      &  mix    &  Large (860k)    &  Large (860k)    &  15.77  &  30.71  &        27.29\\
T              &  L      &  Large (860k)    &  20k+860k        &  19.45  &  29.42  &        27.41\\
T+C CNG03      &  L      &  Large (860k)    &  20k+860k        &  14.22  &  30.33  &        27.48\\
T+C CNG03      &  s      &  Baseline (20k)  &  Baseline (20k)  &  23.27  &  30.75  &        27.62\\
T+C full tags  &  L      &  Large (860k)    &  20k+860k        &  14.06  &  30.64  &  \max{28.12}\\
\end{tabular}
%\end{center}
\caption{The effect of additional data in T and T+C scenarios.}
\label{moredatachart}
\end{figure}


Figure \ref{moredatachart} gives a chart and full details on our experiments
with adding more data into the T and T+C scenarios. We varied the scenario (T or
T+C), the level of detail in the T+C scenario (full tags vs. CNG03), the size of the
parallel corpus used to extract phrases (Baseline vs. Large, as described in
section \ref{baselinelargecorpus}) and the size or combination of target side language models (a
single LM based on the Baseline or Large corpus, or both of them with separate
weights set in the MERT training).

Several observations can be made:

\begin{itemize}

\item Ignoring the domain difference and using only the single Large language model
  (denoted \equo{mix} in the chart) hurts. Only the \equo{T+C CNG03} scenario
  does not confirm this observation and we believe this can be attributed to
  some randomness in MERT training of \equo{T+C CNG03 s$^+$}.

\item CNG03 outperforms full tags only in small data setting, with large data
  (treating the domain difference properly), full tags are significantly better.

\item The improvement of T+C over T decreases if we use more data.
\end{itemize}




\subsubsection{First Experiments with Verb Frames}



\subsubsubsection{Microstudy: Current MT Errors}

The previous sections described significant improvements gained on small data
sets when checking morphological agreement or adding more data (BLEU raised from
25.82\% to 27.62\% or up to 28.12\% with additional out-of-domain parallel
data). However, the best result achieved is still far below the margin of
lemmatized BLEU, as estimated in section \ref{margin}. In fact, lemmatized BLEU
of our best result is yet a bit higher (35\%), indicating that T+C 
improve not only morphology, but also word order or lexical selection issues.

We performed a microstudy on local agreement between adjectives and their
governing nouns. Altogether 74\% of adjectives agreed with the governing noun
and only 2\% of adjectives did not agree; the full listing is given in Table
\ref{agreement}.

\begin{table}[t]
\begin{center}
\begin{tabular}{lr}
\bf An adjective in MT output\dots  &  \llap{\bf Portion}\\
\hline
\bf agrees with the governing noun  &  \bf 74\%\\
depends on a verb (cannot check the agreement)  &  7\%\\
misses the governing noun (cannot check the agreement)  &  7\%\\
should not occur in MT output  &  5\%\\
has the governing noun not translated (cannot check the agreement)  &  5\%\\
\bf mismatches with the governing noun  &  \bf 2\%\\
\end{tabular}
\end{center}
\caption{Microstudy: adjectives in \tocs{} MT output.}
\label{agreement}
\end{table}


Local agreement thus seems to be relatively correct. In order to find the source
of the remaining morphological errors, we performed another microstudy of
current best MT output (BLEU 28.12\%) using an intuitive metric. We checked
whether Verb-Modifier relations are properly preserved during the translation of
15 sample sentences.


%Verb:   
%ok      43 (55.8 %)
%miss    21 (27.3 %)
%bad     11 (14.3 %)
%semi    2 (2.6 %)
%Celkem  77 (100.0 %)
%
%Noun    
%ok      61 (79.2 %)
%bad     9 (11.7 %)
%unk     4 (5.2 %)
%semi    2 (2.6 %)
%miss    1 (1.3 %)
%Celkem  77 (100.0 %)


The \emph{source} text of the sample sentences contained 77
Verb-Modifier pairs. Table \ref{verbmod} lists our observations on the two
members in each Verb-Modifier pair. We see that only 43\% of verbs are
translated correctly and 79\% of nouns are translated correctly. The system
tends to skip verbs quite often (21\% of cases).

\begin{table}[t]
\begin{center}
\begin{tabular}{lcc}
\bf Translation of            &  \bf Verb  &  \bf Modifier\\
\hline
\dots{}preserves meaning  &  56\%  &  79\%\\
\dots{}is disrupted       &  14\%  &  12\%\\
\dots{}is missing         &  27\%  &  1\%\\
\dots{}is unknown (not translated)         &  0\%   &  5\%\\
\end{tabular}
\end{center}
\caption{Analysis of 77 Verb-Modifier pairs in 15 sample sentences.}
\label{verbmod}
\end{table}


More
importantly, our analysis has shown that even in cases where both the Verb and
the Modifier are correct, the relation between them in Czech is either
non-gramatical or meaning-disturbing in 56\% of these cases. Commented samples
of such errors are given in Figure \ref{sampleerrors}. The first sample shows
that a strong language model can lead to the choice of a grammatical relation
that nevertheless does not convey the original meaning. The second sample
illustrates a situation where two correct options are available but the
system chooses an inappropriate relation, most probably because of backing off to
a generic pattern verb-noun$_{plural}^{accusative}$. This pattern
is quite common for
for expressing the object role of
many verbs (such as \text{vydat}, see Correct
option 2 in Figure \ref{sampleerrors}), but does not fit well with the verb
\text{vyb\v{e}hnout}. While the target-side data
may be rich enough to learn the generalization vyb\v{e}hnout--s--{\it instr},
no
such generalization is possible with language models over word forms or
morphological tags only. The
target side data will be hardly ever rich enough to learn this particular
structure in all correct morphological and lexical variants:
\text{vyb\v{e}hl--s--reklamou, vyb\v{e}hla--s--reklamami,
vyb\v{e}hl--s--prohl\'{a}\v{s}en\'{\i}m, vyb\v{e}hli--s--ozn\'{a}men\'{\i}m,
\dots}. We would need a 
mixed
model that combines verb lemmas, prepositions and case information to properly
capture the relations.


\begin{figure}

\begin{center}
{\small
\begin{tabular}{|ll|}
\hline
Input:      &  {Keep on investing.}\\
MT output:  &  Pokra\v{c}ovalo investov\'{a}n\'{\i}. (grammar correct here!)\\
Gloss:      &  Continued investing. (Meaning: The investing continued.)\\
Correct:    &  {Pokra\v{c}ujte v investov\'{a}n\'{\i}.}\\
\hline
\end{tabular}
}
\end{center}


\begin{center}
{
\small
\begin{tabular}{|lllll|}
\hline
Input:  &  \multicolumn{4}{l|}{brokerage firms rushed out ads \dots}\\
MT Output:  &  brokersk\'{e}  &  firmy  &  vyb\v{e}hl  &  reklamy\\
Gloss:  &  brokerage  &  firms$_{pl.fem}$  &  ran$_{sg.masc}$  &  ads$_{pl.nom,pl.acc}^{pl.voc,sg.gen}$\\
Correct option 1:  &  brokersk\'{e}  &  firmy  &  vyb\v{e}hly  &  s reklamami$_{pl.instr}$\\
Correct option 2:  &  brokersk\'{e}  &  firmy  &  vydaly  &  reklamy$_{pl.acc}$\\
\hline
\end{tabular}
}
\end{center}
\caption{Two sample errors in translating Verb-Modifier relation from English to
Czech.}
\label{sampleerrors}
\end{figure}







To sum up, the analysis has revealed that in our best MT output:

\begin{itemize}

\item noun-adjective agreement is already quite fine,

\item verbs are often missing,

\item verb-modifier relations are often malformed.
\end{itemize}


\subsubsubsection{Design of Verb Frame Factor}


In this section we describe a model that combines verb lemmas, prepositions and
noun cases to improve the verb-modifier relations on the target side and
possibly to favour keeping verbs in MT output. The model
is incorporated to our MT system in the most simple fashion: we simply create an
additional output factor to explicitly model target verb
valency/subcategorization, i.e. to mark verbs and their modifiers in the output.
An independent language model is used to ensure coherence in the verb frame
factor.

Figure \ref{samplevf} illustrates the process of converting a Czech sentence to
the corresponding verb frame factor. We make use of the dependency analysis of
the sentence and associate each input word with a token:

\begin{itemize}

\item tokens for verbs have the form \text{H:V\ang{subpart of speech}:\ang{verb
  lemma}} indicating that the verb is the head of the frame,

\item tokens for words depending on a verb have the form \text{M:\ang{slot
  description}} to denote verb frame members. Slot description is based on the
  respective modifier:
\begin{itemize}

\item slot description for nouns, pronouns and (nominalized) adjectives contains
  only the case information (e.g. \text{subst$_{nom}$} for nouns or pronouns in
  nominative),

\item slot description for prepositions contains the preposition lemma and the
  case (e.g. \text{prep:na$_{acc}$} for the preposition \text{na} in accusative),

\item sub-ordinating conjunctions are represented by their lemma (e.g.
  \text{subord:zda} for the conjunction \text{zda}),

\item co-ordinating conjunctions are treated in an oversimplified manner, the slot
  description just notes that there was a co-ordinating conjunction. No
  information is propagated from the co-ordinated elements.

\item adverbs are completely ignored, i.e. get a dummy token \text{---}
\end{itemize}

\item punctuation symbols have the form \text{PUNCT:\ang{punctuation symbol}} and
  conjunctions have the form \text{DELIM:\ang{subpart of
  speech}:\ang{conjunction lemma}} to
  keep track of structural delimiters in the verb frame factor,

\item all other words get a dummy token \text{---}.
\end{itemize}

Thus for the beginning of the sample sentence in Figure \ref{samplevf}
\text{Popt\'{a}vka trvale stoup\'{a} za podpory\dots (The demand has been consistently
growing under the encouragement\dots)} we create the following stream:

\begin{center}
M:subst$_{nom}$ --- H:VB:stoupat M:prep:za$_{gen}$ ---
\end{center}

The stream indicates that the verb \text{stoupat} tends to be modified by a
(preceding) subject and (following) argument or adjunct governed by the
preposition \text{za} in genitive.

Keeping in mind the valency theory for Czech (e.g. \perscite{panevova:94}),
there are several limitations in our model:

\begin{itemize}

\item We do not make any distinctions between
argument and adjuncts (except for the above mentioned deletion of adverbs).
Ideally, all adjuncts would get the dummy token \text{---}.


\item In theory,
the order of verb frame members is not grammatically significant for some
languages, so we should allow independent reordering of the verb frame factor.


\item If a verb depends on a verb, their modifiers can be nearly arbitrarily mixed
  within the clause (in Czech). Our model does not distinguish which modifiers
  belong to which of the verbs.

\end{itemize}

Another problem with the verb frame factor is the explicit representation of
the number of intervening words (tokens \text{---}). A skipping language model
would be necessary to describe the linguistic reality more adequately.


%<s id="wsj-underscore-1200.cz:3">
%<f>Popt\'{a}vka<MDl src="a">popt\'{a}vka<MDt src="a">NNFS1-----A----<A>Sb<r>1<g>3
%<f>trvale<MDl src="a">trvale-underscore--caret--underscore-*1\'{y}-underscore-<MDt src="a">Dg-------1A----<A>Adv<r>2<g>3
%<f>stoup\'{a}<MDl src="a">stoupat-underscore-:T<MDt src="a">VB-S---3P-AA---<A>Obj<r>3<g>9
%<f>za<MDl src="a">za-1<MDt src="a">RR--2----------<A>AuxP<r>4<g>3
%<f>podpory<MDl src="a">podpora-underscore--caret--underscore-pen\'{\i}ze;-underscore-ty\v{c};-underscore-mor\'{a}ln\'{\i}-underscore-p.-underscore-<MDt src="a">NNFS2-----A----<A>Adv<r>5<g>4
%<f>vl\'{a}dn\'{\i}<MDl src="a">vl\'{a}dn\'{\i}<MDt src="a">AAFS2----1A----<A>Atr<r>6<g>7
%<f>politiky<MDl src="a">politika-underscore--caret--underscore-v\v{e}da-underscore-<MDt src="a">NNFS2-----A----<A>Atr<r>7<g>5
%<f>,<MDl src="a">,<MDt src="a">Z:-------------<A>AuxX<r>8<g>3
%<f>\v{r}ekl<MDl src="a">\v{r}\'{\i}ci<MDt src="a">VpYS---XR-AA---<A>Pred<r>9<g>0
%<f>mluv\v{c}\'{\i}<MDl src="a">mluv\v{c}\'{\i}<MDt src="a">NNMS1-----A----<A>Sb<r>10<g>9

%poptávka trvale stoupá za podpory prospotřebitelské vládní politiky , řekl mluvčí asociace .
%MEMBER:subst+1 --- HEAD:VB:stoupat_:T MEMBER:Rza-1+2 --- --- --- --- MEMBER:Z- HEAD:Vp:říci MEMBER:subst+1 --- PUNCT:.

%<f>\broken{Popt\'{a}vka\\MEMBER:subst+1}<A>Sb<r>1<g>3
%<f>\broken{trvale\\---}<A>Adv<r>2<g>3
%<f>\broken{stoup\'{a}\\HEAD:VB:stoupat\_:T}<A>Obj<r>3<g>9
%<f>\broken{za\\MEMBER:Rza-1+2}<A>AuxP<r>4<g>3
%<f>\broken{podpory\\---}<A>Adv<r>5<g>4
%<f>\broken{vl\'{a}dn\'{\i}\\---}<A>Atr<r>6<g>7
%<f>\broken{politiky\\---}<A>Atr<r>7<g>5
%<f>\broken{,\\MEMBER:Z-}<A>AuxX<r>8<g>3
%<f>\broken{\v{r}ekl\\HEAD:Vp:říci}<A>Pred<r>9<g>0
%<f>\broken{mluv\v{c}\'{\i}\\MEMBER:subst+1}<A>Sb<r>10<g>9
%<f>\broken{.\\PUNCT:.}<A>AuxK<r>11<g>0

\begin{figure}[ht]
\PSforPDF{
\def\ctr#1{\hbox to \wd0{\hss{}\hbox{\strut{}#1}\hss{}}}
\def\ctrclap#1{\hbox to \wd0{\hss{}\strut\clap{#1}\hss{}}}
\def\brok#1#2#3#4{{\setbox0=\hbox{#1}\vbox{\ctr{#1}\ctrclap{#3}\ctr{#4}\ctr{\scriptsize #2}}}}

\begin{idtree}
\node{1}{3}{\vbox{\brok{Popt\'{a}vka}{Demand}{}{M:subst$_{nom}$}}}
\node{2}{3}{\brok{trvale}{consistently}{}{---}{}}
\node{3}{2}{\brok{stoup\'{a}}{grows}{H:VB:stoupat}{}}
\node{4}{3}{\brok{za}{\lower0.4ex\hbox{under}}{}{M:prep:za$_{gen}$}}
\node{5}{4}{\brok{podpory}{\raise1mm\hbox{encouragement}}{---}{}}
\node{6}{6}{\brok{vl\'{a}dn\'{\i}}{\lower0.4ex\hbox{government}}{---}{}}
\node{7}{5}{\brok{politiky}{policies}{---}{}}
\node{8}{3}{\brok{,}{,}{}{M:Z-}}
\node{9}{1}{\brok{\v{r}ekl}{said}{H:Vp:\v{r}\'{\i}ci}{}}
\node{10}{2}{\brok{mluv\v{c}\'{\i}}{spokesman}{}{M:subst$_{nom}$}}
\node{11}{1}{\brok{.}{.}{PUNCT:.}{}}
\edge{3}{1}{Sb  }
\edge{3}{2}{Adv}
\edge{9}{3}{Obj}
\edge{3}{4}{AuxP}
\edge{4}{5}{Adv}
\edge{7}{6}{Atr}
\edge{5}{7}{Atr}
\edge{3}{8}{AuxX}
\edge{9}{10}{Sb}
\end{idtree}
}
\caption{Verb frame factor based on dependency syntax tree of a sample sentence:
\text{
Demand has been growing consistently under the encouragement of government policies, a spokesman said.
}}
\label{samplevf}
\end{figure}

%Popt\'{a}vka   	Sb  	 1	3	popt\'{a}vka	MEMBER:subst+1
%trvale         	Adv 	 2	3	trvale  	---
%stoup\'{a}     	Obj 	 3	9	stoup\'{a}  	HEAD:VB:stoupat\_:T
%za             	AuxP	 4	3	za      	MEMBER:Rza-1+2
%podpory        	Adv 	 5	4	podpory 	---
%vl\'{a}dn\'{\i}	Atr 	 6	7	vl\'{a}dn\'{\i}  	---
%politiky       	Atr 	 7	5	politiky	---
%,              	AuxX	 8	3	,       	MEMBER:Z-
%\v{r}ekl       	Pred	 9	0	\v{r}ekl    	HEAD:Vp:\v{r}\'{\i}ci
%mluv\v{c}\'{\i}	Sb  	10	9	mluv\v{c}\'{\i}  	MEMBER:subst+1
%.              	AuxK	11	0	.       	PUNCT:.

%<s id="wsj-underscore-1200.mrg:3">
%<f>Demand<l>demand<t>NN<A>Sb<r>1<g>4<x name="Wf">-SBJ<x name="Wl">NP~-SBJ<x name="wsj-underscore-id">wsj-underscore-1200.mrg:3/1<x name="wsj-underscore-id2">wsj-underscore-1200.mrg:3/1
%<f>has<l>have<t>VBZ<A>AuxV<r>2<g>4<x name="wsj-underscore-id">wsj-underscore-1200.mrg:3/3<x name="wsj-underscore-id2">wsj-underscore-1200.mrg:3/2
%<f>been<l>be<t>VBN<A>Atr<r>3<g>4<x name="wsj-underscore-id">wsj-underscore-1200.mrg:3/4<x name="wsj-underscore-id2">wsj-underscore-1200.mrg:3/3
%<f>growing<l>grow<t>VBG<A>Obj<r>4<g>17<x name="Wf">-TPC-1<x name="Wl">S~-TPC-1<x name="wsj-underscore-id">wsj-underscore-1200.mrg:3/5<x name="wsj-underscore-id2">wsj-underscore-1200.mrg:3/4
%<f>consistently<l>consistently<t>RB<A>Adv<r>5<g>4<x name="Wf">-MNR<x name="Wl">ADVP-MNR<x name="wsj-underscore-id">wsj-underscore-1200.mrg:3/6<x name="wsj-underscore-id2">wsj-underscore-1200.mrg:3/5
%<f>under<l>under<t>IN<A>AuxP<r>6<g>4<x name="Wf">-LOC<x name="Wl">PP-LOC<x name="wsj-underscore-id">wsj-underscore-1200.mrg:3/8<x name="wsj-underscore-id2">wsj-underscore-1200.mrg:3/6
%<f>the<l>the<t>DT<A>Atr<r>7<g>8<x name="wsj-underscore-id">wsj-underscore-1200.mrg:3/9<x name="wsj-underscore-id2">wsj-underscore-1200.mrg:3/7
%<f>encouragement<l>encouragement<t>NN<A>Adv<r>8<g>6<x name="Wl">NP~<x name="wsj-underscore-id">wsj-underscore-1200.mrg:3/10<x name="wsj-underscore-id2">wsj-underscore-1200.mrg:3/8
%<f>of<l>of<t>IN<A>AuxP<r>9<g>8<x name="Wl">PP<x name="wsj-underscore-id">wsj-underscore-1200.mrg:3/12<x name="wsj-underscore-id2">wsj-underscore-1200.mrg:3/9
%<f>pro-consumption<l>pro-consumption<t>NN<A>Atr<r>10<g>12<x name="wsj-underscore-id">wsj-underscore-1200.mrg:3/13<x name="wsj-underscore-id2">wsj-underscore-1200.mrg:3/10
%<f>government<l>government<t>NN<A>Atr<r>11<g>12<x name="wsj-underscore-id">wsj-underscore-1200.mrg:3/14<x name="wsj-underscore-id2">wsj-underscore-1200.mrg:3/11
%<f>policies<l>policy<t>NNS<A>Atr<r>12<g>9<x name="Wl">NP~<x name="wsj-underscore-id">wsj-underscore-1200.mrg:3/15<x name="wsj-underscore-id2">wsj-underscore-1200.mrg:3/12
%<D>
%<d>,<l>,<t>,<A>AuxX<r>13<g>17<x name="wsj-underscore-id">wsj-underscore-1200.mrg:3/24<x name="wsj-underscore-id2">wsj-underscore-1200.mrg:3/13
%<f>an<l>an<t>DT<A>Atr<r>14<g>16<x name="wsj-underscore-id">wsj-underscore-1200.mrg:3/25<x name="wsj-underscore-id2">wsj-underscore-1200.mrg:3/14
%<f>association<l>association<t>NN<A>Atr<r>15<g>16<x name="wsj-underscore-id">wsj-underscore-1200.mrg:3/26<x name="wsj-underscore-id2">wsj-underscore-1200.mrg:3/15
%<f>spokesman<l>spokesman<t>NN<A>Sb<r>16<g>17<x name="Wf">-SBJ<x name="Wl">NP~-SBJ<x name="wsj-underscore-id">wsj-underscore-1200.mrg:3/27<x name="wsj-underscore-id2">wsj-underscore-1200.mrg:3/16
%<f>said<l>say<t>VBD<A>Pred<r>17<g>0<x name="Wl">S<x name="wsj-underscore-id">wsj-underscore-1200.mrg:3/29<x name="wsj-underscore-id2">wsj-underscore-1200.mrg:3/17
%<D>
%<d>.<l>.<t>.<A>AuxK<r>18<g>0<x name="wsj-underscore-id">wsj-underscore-1200.mrg:3/35<x name="wsj-underscore-id2">wsj-underscore-1200.mrg:3/18


\subsubsubsection{Preliminary Results with Verb Frame Factor}

Table \ref{vfresults} displays BLEU scores of the scenario translate-and-check verb factor
(T+Cvf) compared to the single-factored baseline (T). Word alignment for these
experiments was obtained using grow-diag-final heuristic on stemmed English and
lemmatized Czech texts. Only the baseline corpus (20k sentences) was used to
extract phrase tables. The verb frame factor language model is a simple $n$-gram
LM with $n$ of 7, 9 or 11 and it is based either on the baseline corpus (PCEDT)
or the Czech side of the Large corpus. In all cases, a simple trigram model
checks the fluency of word form stream.


\begin{table}[ht]
\begin{center}
\begin{tabular}{lccc}
                 &  BLEU.dev.opt  &  BLEU.dev.std  &  BLEU.opt\\
\hline
T+Cvf LM-11gr-Large  &  28.68        &  19.51         &  24.23\\
T+Cvf LM-7gr-Baseline   &  28.54        &  19.75         &  25.05\\
T+Cvf LM-7gr-Large   &  28.32        &  19.69         &  25.07\\
T+Cvf LM-9gr-Large   &  27.98        &  19.55         &  25.09\\
Baseline: T  &  29.24        &  25.68        &  \max{25.23}\\
\end{tabular}
\end{center}
\caption{Preliminary results with checking of verb frame factor.}
\label{vfresults}
\end{table}

Unfortunately, all T+Cvf results fall below the single-factored baseline.
A more
thorough analysis of the data available and the hypothesis space would be
necessary to explain why our verb frame factor tends to mislead the decoder.

XXX why did it fail? I'm going to check if at least the verb-mod-relations get
improved.


\subsubsection{Single-factored Results \toen{}}

Our primary interest was in \tocs{} translation but we also experimented with
\toen{} direction, mainly to allow for comparison with previous reported
results.

It should be noted that translating to English in our setting is easier. In
general, there are fewer word forms in English so language models face milder
data spareness and there are fewer chances to make an error (BLEU would notice).
Moreover, the particular test set we use contains input Czech text that came
from an English original and was translated sentence by sentence.  The Czech
thus probably does not exhibit full richness and complexity of word order and
language constructions and is easier to translate back to English than a generic
Czech text would be.


\begin{table}[ht]
\begin{center}
\small
\begin{tabular}{lccccc}
Scenario  &  Parallel Corpus  &  Language Models                &  Dev (std)  &  Dev (opt)  &  Test (opt)\\
\hline
T         &  Baseline (20k)   &  Baseline (20k)                 &      28.97  &      35.39  &        28.50\\
T+C       &  Baseline (20k)   &  Baseline (20k)                 &      23.07  &      36.13  &        28.66\\
T         &  Large (860k)     &  20k+860k                       &      19.31  &      39.60  &        33.37\\
T         &  Large (860k)     &  \clap{Large (860k, i.e. mix)}  &      28.94  &      40.15  &  \max{34.12}\\
\end{tabular}
\end{center}
\caption{Sample \toen{} BLEU scores.}
\label{english}
\end{table}

Table \ref{english} lists Moses results of \toen{} translation. We observe a
minor improvement when checking the part-of-speech factor (T+C). A larger
improvement is obtained
by adding more data and quite differently from \tocs{} results (see section
\ref{moredata}), mixing in-domain
and out-of-domain LM data does not hurt the performance.




\subsubsection{Summary and Conclusion}

We experimented with factored \tocs{} translation. The majority of our
experiments were carried out in a small data setting and we translated to a
morphologically rich language. In this setting, lemmatization or stemming of
training data is vital for obtaining reliable alignments. Multi-factored
translation for ensuring coherence of morphological properties of output words
significantly increases BLEU performance, although the effect is reduced with
additional training data. Experiments also indicate that more complex
translation scenarios lower the scores, probably due to more severe search
errors.

Our \tocs{} experiments confirm that in minimum-error-rate training, it is
helpful to keep language models based on in- and out-of-domain data separate. We
did not observe this domain sensitivity in \toen{} direction.

Based on manual analysis of sample output sentences, we also conducted some
preliminary experiments on using target-side syntactic information in order to
improve grammaticality of verb-modifier relations.  The results are rather
inconclusive and further refinement of the model would be necessary.



\subsection{Acknowledgement}

The work on this project was partially supported by the grants 
Collegium Informaticum GA\v{C}R 201/05/H014,
% doktorandsky
Grant No. 0530118 of the National Science Foundation of the USA,
% PIRE anglicky
and grants No. ME838 and GA405/06/0589.
% PIRE cesky.
The translation of reference sentences was supported by the grants
NSF 0122466 and M\v{S}MT \v{C}R 1P05ME786. The collection of additional training
data was supported by the grant GAUK 351/2005.
Any opinions, findings and conclusions or recommendations expressed in this
material are those of the authors and do not necessarily reflect the views of
the respective grant agencies.

I would like to express my gratitude to all members in our team for excellent
atmosphere and very pleasant collaboration. I also wish to thank specifically to
the leader of our team, Philipp Koehn, and the teachers Wade Shen and Marcello
Federico for full support and many stimulating comments and suggestions.
Finally, I cannot forget to mention the Center for Language and Speech Processing at Johns Hopkins
University for making this workshop possible.


}  % wrapping all Ondrej's content to prevent confusing macros


\section{Chinese-English}
{\sc Wade Shen}

\section{Confusion Network Decoding}
{\sc Wade Shen and Richard Zens}

\section{Tuning}
{\sc Nicola Bertoldi}

\section{Linguistic Information for Word Alignment}
%{\sc Alexandra Constantin}

\subsection{Word Alignment\\}

If we open a common bilingual dictionary, we may find an entry
like\\
\begin{center}
\textbf{Haus} = house, building, home, household\\
\end{center}
Many words have multiple translations, some of which are more likely
than others.

If we had a large collection of German text, paired with its
translation into English, we could count how often $Haus$ is
translated into each of the given choices. We can use the counts to
estimate a lexical translation probability distribution

\begin{center}
$t : e|f \rightarrow t(e|f)$
\end{center}

that, given a foreign word, $f$, returns a probability for each
choice for an English translation $e$, that indicates how likely
that translation is.


We can derive an estimate of the translation probability
distribution from the data by using the ratio of the counts. For
example, if we have $10000$ occurrences of $Haus$ and $8000$
translate to $house$, then $t(house|Haus)=0.8$.

For some words that are infrequent in the corpus, the estimates of
the probability distribution are not very accurate. Using other
linguistic information, such as observing that in a specific
language pair verbs usually get translated as verbs, could help in
building a more accurate translation.

Let's look at an example. Imagine we wanted to translate the German
sentence \textbf{\emph{das Haus ist klein}}. The sentence can be
translated word by word into English. One possible translation is
\textbf{\emph{the house is small}}.


Implicit in these translations is an alignment, a mapping from
German words to English words:

\includegraphics[viewport = 100 440 400 530,clip]{constantin-figure1.pdf}

An alignment can be formalized with an alignment function $a : i
\rightarrow j$. This function maps each English target word at
position $i$ to a German source word at position $j$.

For example, if we are given the following pair of sentences:

\includegraphics[viewport = 100 400 400 550,clip]{constantin-figure2.pdf}

the alignment function will be

$a:\{1 \rightarrow 3, 2 \rightarrow 4, 3 \rightarrow 2, 4
\rightarrow 1\}$.


\subsection{IBM Model 1\\}

Lexical translations and the notion of alignment allow us to define
a model capable of generating a number of different translations for
a sentence, each with a different probability. One such model is IBM
Model 1, which will be described below.

For each target word $e$ that is produced by the model from a source
word $f$, we want to factor in the translation probability $t(e|f)$.

The translation probability of a foreign sentence
$\textbf{f}=(f_1,\dots,f_{l_f})$ of length $l_f$ into an English
sentence $\textbf{e}=(e_1,\dots, e_{l_e})$ of length $l_e$ with an
alignment of each English word $e_j$ to a foreign word $f_i$
according to alignment $a:j \rightarrow i$ is:

\begin{center}
$p(\textbf{e},a|\textbf{f}) = \frac{\epsilon}{(l_f +
1)^{l_e}}\prod_{j=1}^{l_e} t(e_j|f_{a(j)})$
\end{center}


\subsection{Learning the Lexical Translation Model}

A method for estimating these translation probability distributions
from sentence-aligned parallel text is now needed.

The previous section describes a strategy for estimating the lexical
translation probability distributions from a word-aligned parallel
corpus. However, while large amounts of sentence-aligned parallel
texts can be easily collected, word-aligned data cannot. We would
thus like to estimate these lexical translation distributions
without knowing the actual word alignment, which we consider a
hidden variable. To do this, we use the Expectation-Maximization
algorithm:

\paragraph{EM algorithm}
\begin{itemize}
\item{Initialize model (typically with uniform distribution)}
\item{Apply the model to the data (expectation step)}
\item{Learn the model from the data (maximization step)}
\item{Iterate steps 2-3 until convergence}
\end{itemize}

First, we initialize the model. Without prior knowledge, uniform
probability distributions are a good starting point. In the
expectation step, we apply the model to the data and estimate the
most likely alignments. In the maximization step, we learn the model
from the data and augment the data with guesses for the gaps.

 \textbf{Expectation step\\}

When we apply the model to the data, we need to compute the
probability of different alignments given a sentence pair in the
data:

\begin{center}
$p(a|\textbf{e},\textbf{f}) =
\frac{p(\textbf{e},a|\textbf{f})}{p(\textbf{e}|\textbf{f})}$
\end{center}

$p(\textbf{e}|\textbf{f})$, the probability of translating sentence
$\textbf{f}$ into sentence $\textbf{e}$ is derived as:

\begin{center}

$p(\textbf{e}|\textbf{f}) = \sum_a p(\textbf{e},a|\textbf{f}) =
\prod_{j=1}^{l_e} \sum_{i=0}^{l_f}t(e_j|f_i)$
\end{center}

Putting the previous two equations together,

\begin{center}

 $p(a|\textbf{e},\textbf{f}) =
\frac{p(\textbf{e},a|\textbf{f})}{p(\textbf{e}|\textbf{f})}
=\prod_{j=1}^{l_e} \frac {t(e_j | f_{a(j)})}{\sum_{i=0}^{l_f}
t(e_j|f_i)}$.


\end{center}

\textbf{Maximization Step}

For the maximization step, we need to collect counts over all
possible alignments, weighted by their probabilities. For this
purpose, we define a count function $c$ that collects evidence from
a sentence pair $(\textbf{e},\textbf{f})$ that a particular source
word $f$  translates into a particular target word $e$.

\begin{center}

$c(e|f;\textbf{e},\textbf{f}) = \sum_a p(a|\textbf{e},\textbf{f}) =
\frac{t(e|f)}{\sum_{j=1}^{l_e}t(e|f_{a(j)})} \sum_{j=1}^{l_e}
\delta(e,e_j)\sum_{i=0}^{l_f}\delta(f,f_i)$

\end{center}

where $\delta(x,y)$ is $1$ is $x=y$ and $0$ otherwise.

Given the count function, we can estimate the new translation
probability distribution by:

\begin{center}
$t(e|f;\textbf{e},\textbf{f}) =
\frac{\sum_{(\textbf{e},\textbf{f})}c(e|f;\textbf{e},\textbf{f})}{\sum_f\sum_{(\textbf{e},\textbf{f})}c(e|f;\textbf{e},\textbf{f})}$.
\end{center}


\subsection{Introducing Part of Speech Information to the Model}
In order to introduce part of speech information to the model, we
need to consider the probability of translating a foreign word
$f_{word}$ with part of speech $f_{POS}$ into English word
$e_{word}$ with part of speech $e_{POS}$. In order words, we need to
consider the translation probability distribution $t(e|f)$, where
$e$ and $f$ are vectors, $e = (e_{word},e_{POS})$, $f = (f_{word},
f_{POS})$. In order to estimate this density function, we need to
make some independence assumptions. Depending on the independence
assumption, several models can be formed:

\paragraph{POS Model 1}

Assuming that words are independent from their parts of speech, we
can estimate the translation density as:

\begin{center}
$t(e|f) = t(e_{word}|f_{word}) * t(e_{POS}|f_{POS})$
\end{center}



\paragraph{POS Model 2}

Making weaker independence assumption, the translation density can
be estimated as:

\begin{center}

$t(e|f) = \lambda p(e_{POS}|e_{word})t(e_{word}|f_{word}) +
(1-\lambda) p(e_{word}|e_{POS}) t(e_{POS}|f_{POS})$
\end{center}

This model has the advantage that it can weigh the importance given
to part-of-speech information.

\subsection{Experiment}
To test whether part-of-speech information improves alignment
quality, we compared alignments generated using IBM Model 1,
alignments generated using only part-of-speech information, and
alignments generated using POS Model 1 against manual alignments.
The metric used to compare the alignments was $AER$ (alignment error
rate). The data consisted of European Parliament German and English
parallel corpora. Experiments were done using different sizes of
corpora. The scores are presented in the following table:

\includegraphics[viewport = 90 600 510 730,clip]{constantin-table.pdf}

The first row indicates the number of sentences used for training
and the first column indicates the model used to generate
alignments.

As expected, the $AER$ of the alignments generated using only part
of speech information are very high, indicating that part-of-speech
information is not sufficient to generate good alignments. However,
an $AER$ of around $.75$ indicates that there is some information
provided by part-of-speech information that could be useful.

The $AER$ of alignments generated with IBM Model 1 doesn't
statistically differ from the $AER$ of alignments generated with the
additional part of speech information. One reason for this might be
that the part-of-speech probability was given equal weight to the
word probability, even though the latter is more important. POS
Model 2 might thus generate an improvement in $AER$.



\chapter{Conclusions}
{\sc Philipp Koehn: Accomplishments}

\appendix

\chapter{Follow-Up Research Proposals}

\section{Translation with syntax and factors: Handling global and local dependencies in SMT}
{\sc Brooke Cowan: just cut and paste your proposal here}

\section{Exploiting Ambiguous Input in Statistical Machine Translation}
{\sc Richard Zens: just cut and paste your proposal here}

\bibliographystyle{apalike}
\bibliography{mt}

\printindex

\end{document}
